%-------------------------
% Resume in Latex
% Author : Sidratul Muntaha Ahmed
% License : MIT
%------------------------

\documentclass[letterpaper,11pt]{article}

\usepackage{latexsym}
\usepackage[empty]{fullpage}
\usepackage{titlesec}
\usepackage{marvosym}
\usepackage[usenames,dvipsnames]{color}
\usepackage{verbatim}
\usepackage{enumitem}
\usepackage[hidelinks]{hyperref}
\usepackage{fancyhdr}
\usepackage[english,russian]{babel}
\usepackage{tabularx}
\input{glyphtounicode}



%----------FONT OPTIONS----------
% sans-serif
% \usepackage[sfdefault]{FiraSans}
% \usepackage[sfdefault]{roboto}
% \usepackage[sfdefault]{noto-sans}
% \usepackage[default]{sourcesanspro}

% serif
% \usepackage{CormorantGaramond}
% \usepackage{charter}


\pagestyle{fancy}
\fancyhf{} % clear all header and footer fields
\fancyfoot{}
\renewcommand{\headrulewidth}{0pt}
\renewcommand{\footrulewidth}{0pt}

% Adjust margins
\addtolength{\oddsidemargin}{-0.5in}
\addtolength{\evensidemargin}{-0.5in}
\addtolength{\textwidth}{1in}
\addtolength{\topmargin}{-.5in}
\addtolength{\textheight}{1.0in}

\urlstyle{same}

\raggedbottom
\raggedright
\setlength{\tabcolsep}{0in}

% Sections formatting
\titleformat{\section}{
  \vspace{-4pt}\scshape\raggedright\large
}{}{0em}{}[\color{black}\titlerule \vspace{-5pt}]

% Ensure that generate pdf is machine readable/ATS parsable
\pdfgentounicode=1

%-------------------------
% Custom commands
\newcommand{\resumeItem}[1]{
  \item\small{
    {#1 \vspace{-2pt}}
  }
}

\newcommand{\resumeSubheading}[4]{
  \vspace{-2pt}\item
    \begin{tabular*}{0.97\textwidth}[t]{l@{\extracolsep{\fill}}r}
      \textbf{#1} & #2 \\
      \textit{\small#3} & \textit{\small #4} \\
    \end{tabular*}\vspace{-7pt}
}

\newcommand{\resumeSubSubheading}[2]{
    \item
    \begin{tabular*}{0.97\textwidth}{l@{\extracolsep{\fill}}r}
      \textit{\small#1} & \textit{\small #2} \\
    \end{tabular*}\vspace{-7pt}
}

\newcommand{\resumeProjectHeading}[2]{
    \item
    \begin{tabular*}{0.97\textwidth}{l@{\extracolsep{\fill}}r}
      \small#1 & #2 \\
    \end{tabular*}\vspace{-7pt}
}

\newcommand{\resumeSubItem}[1]{\resumeItem{#1}\vspace{-4pt}}

\renewcommand\labelitemii{$\vcenter{\hbox{\tiny$\bullet$}}$}

\newcommand{\resumeSubHeadingListStart}{\begin{itemize}[leftmargin=0.15in, label={}]}
\newcommand{\resumeSubHeadingListEnd}{\end{itemize}}
\newcommand{\resumeItemListStart}{\begin{itemize}}
\newcommand{\resumeItemListEnd}{\end{itemize}\vspace{-5pt}}

%-------------------------------------------
%%%%%%  RESUME STARTS HERE  %%%%%%%%%%%%%%%%%%%%%%%%%%%%


\begin{document}

%----------HEADING----------


\begin{center}
    \textbf{\Huge \scshape Дениченко Александр} \\ \vspace{1pt}
    \small 8-921-512-43-61 $|$ \href{mailto:a.deni444enko@mail.ru}{\underline{a.deni444enko@mail.ru}}
\end{center}


%-----------EDUCATION-----------
\section{Education}
  \resumeSubHeadingListStart
    \resumeSubheading
      {Национальный исследовательский университет ИТМО}{Санкт-Петербург, RUS}
      {Бакалавриат (Высшее)}{Окончание - 2026 год}
          \resumeItemListStart
            \resumeItem{Вычислительная техника, Системное прикладное и программное обеспечение}
            \resumeItem{Средний балл: 4B-5A}
            
        \resumeItemListEnd

  \resumeSubHeadingListEnd

%-----------EXPERIENCE-----------
\section{Experience}
  \resumeSubHeadingListStart

    \resumeSubheading
      {Java-разработчик (Junior)}{Ноябрь 2022 — Март 2025}
      {Laboratory of Cloud-Oriented Solutions and Modelling}{Лаборатория на базе университета, RUS}
      \resumeItemListStart
        \resumeItem{Оптимизировал выгрузку новых данных для сотрудников путём добавления функции множественного импорта из CSV-файлов.}
        \resumeItem{Сделал распределённые транзакции с собственой реализацией 2PC, которые позволили добавить сохранение файлов импорта в S3-совместимое хранилище.}
        \resumeItem{Отсутствовал механизм, позволяющий разработчикам предлагать идеи, что затрудняло генерацию инноваций. Реализовал систему для добавления идей от разработчиков, модерации их менеджерами и преобразования в задачи. Так же добавил механизмы тегов (включая функции и индексы на уровне базы данных) для быстрого поиска и сортировки по задачам, людям, спринтам, командам, релизам.}
        \resumeItem{Добавил логику работы WebSocket для улучшения взаимодействия в реальном времени.}
        \resumeItem{Реализовал панель администратора с JWT-аутентификацией через Spring Security. Обновил настройки Nginx для устранения CORS-проблем.}
        \resumeItem{Проводил нагрузочные тесты при помощи JMeter, по итогам которых переписал полностью логику транзакций: некоторые транзакциионные операции были переделаны на ручной контроль с использованием JTA, XA. Улучшилась стабильность системы при пиковых нагрузках.}

        \resumeItemListEnd
      
     
    \resumeSubHeadingListEnd

%-----------TECHNICAL SKILLS-----------
\section{Technical Skills}
 \begin{itemize}[leftmargin=0.15in, label={}]
    \small{\item{
     \textbf{Стек}{: Java, Hibernate ORM, Linux, JPA, JTA, PostgreSQL, Spring Framework, Spring, REST API, Hibernate, Spring Boot, Spring MVC, Spring Data, Spring Security, ООП, Git,
     GitHub, Gitlab, Docker, SOLID, Nginx, Spring Web, Jenkins, Maven, Tomcat, SQL, ORM, Kafka, Redis, Jira, Flyway}
    }}
 \end{itemize}

%-------------------------------------------

%-----------COMMUNITY AND LEADERSHIP-----------
\section{Community \& Leadership}
  \resumeSubHeadingListStart

    \resumeSubheading
      {Работа в команде}{}
      {Laboratory of Cloud-Oriented Solutions and Modelling}{}
      \resumeItemListStart
        \resumeItem{Наставничество новых членов команды}
        \resumeItem{Эффективная коммуникация}
        \resumeItem{Обмен опытом/знаниями}
        \resumeItem{Работа по методологии Agile}
      \resumeItemListEnd

    \resumeSubHeadingListEnd

\end{document}
