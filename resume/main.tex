\documentclass[letterpaper,11pt]{article}

\usepackage{latexsym}
\usepackage[empty]{fullpage}
\usepackage{titlesec}
\usepackage{marvosym}
\usepackage[usenames,dvipsnames]{color}
\usepackage{verbatim}
\usepackage{enumitem}
\usepackage[hidelinks]{hyperref}
\usepackage{fancyhdr}
\usepackage[english,russian]{babel}
\usepackage{tabularx}
\input{glyphtounicode}
\usepackage[utf8]{inputenc}
\usepackage[T1,T2A]{fontenc}


%----------FONT OPTIONS----------
% sans-serif
\usepackage[sfdefault]{FiraSans}
\usepackage[sfdefault]{roboto}
\usepackage[sfdefault]{noto-sans}
\usepackage[default]{sourcesanspro}

% serif
\usepackage{CormorantGaramond}
\usepackage{charter}


\pagestyle{fancy}
\fancyhf{} % clear all header and footer fields
\fancyfoot{}
\renewcommand{\headrulewidth}{0pt}
\renewcommand{\footrulewidth}{0pt}

% Adjust margins
\addtolength{\oddsidemargin}{-0.5in}
\addtolength{\evensidemargin}{-0.5in}
\addtolength{\textwidth}{1in}
\addtolength{\topmargin}{-.5in}
\addtolength{\textheight}{1.0in}

\urlstyle{same}

\raggedbottom
\raggedright
\setlength{\tabcolsep}{0in}

% Sections formatting
\titleformat{\section}{
  \vspace{-4pt}\scshape\raggedright\large
}{}{0em}{}[\color{black}\titlerule \vspace{-5pt}]

% Ensure that generate pdf is machine readable/ATS parsable
\pdfgentounicode=1

%-------------------------
% Custom commands
\newcommand{\resumeItem}[1]{
  \item\small{
    {#1 \vspace{-2pt}}
  }
}

\newcommand{\resumeSubheading}[4]{
  \vspace{-2pt}\item
    \begin{tabular*}{0.97\textwidth}[t]{l@{\extracolsep{\fill}}r}
      \textbf{#1} & #2 \\
      \textit{\small#3} & \textit{\small #4} \\
    \end{tabular*}\vspace{-7pt}
}

\newcommand{\resumeSubSubheading}[2]{
    \item
    \begin{tabular*}{0.97\textwidth}{l@{\extracolsep{\fill}}r}
      \textit{\small#1} & \textit{\small #2} \\
    \end{tabular*}\vspace{-7pt}
}

\newcommand{\resumeProjectHeading}[2]{
    \item
    \begin{tabular*}{0.97\textwidth}{l@{\extracolsep{\fill}}r}
      \small#1 & #2 \\
    \end{tabular*}\vspace{-7pt}
}

\newcommand{\resumeSubItem}[1]{\resumeItem{#1}\vspace{-4pt}}

\renewcommand\labelitemii{$\vcenter{\hbox{\tiny$\bullet$}}$}

\newcommand{\resumeSubHeadingListStart}{\begin{itemize}[leftmargin=0.15in, label={}]}
\newcommand{\resumeSubHeadingListEnd}{\end{itemize}}
\newcommand{\resumeItemListStart}{\begin{itemize}}
\newcommand{\resumeItemListEnd}{\end{itemize}\vspace{-5pt}}

%-------------------------------------------
%%%%%%  RESUME STARTS HERE  %%%%%%%%%%%%%%%%%%%%%%%%%%%%


\begin{document}

%----------HEADING----------


\begin{center}
    \textbf{\Huge \scshape Александр Дениченко} \\ \vspace{1pt}
    \small 8-921-512-43-61 $|$ \href{mailto:a.deni444enko@mail.ru}{\underline{a.deni444enko@mail.ru}} $|$ 
    \href{https://t.me/nord_west_zap}{\underline{t.me/nord\_west\_zap}} $|$
    \href{https://github.com/Alex-de-bug}{\underline{github.com/Alex-de-bug}}
\end{center}

%-----------TECHNICAL SKILLS-----------
\section{Технические навыки}
 \begin{itemize}[leftmargin=0.15in, label={}]
    \small{\item{
     \textbf{Языки программирования}{: Java (Core, Collections, Concurrency), C, Python, Assembly (RISC-V)} \\
     \textbf{Фреймворки и технологии}{: Spring (Mail, Boot, MVC, Security, Data JPA), Hibernate, Criteria API} \\
     \textbf{Веб-разработка}{: React, HTML/CSS, RESTful APIs} \\
     \textbf{Инструменты DevOps}{: Docker, Git (GitHub, GitLab), Nginx, Maven, Flyway} \\
     \textbf{Базы данных}{: PostgreSQL, Redis} \\
     \textbf{Безопасность}{: JWT, OAuth 2.0} \\
     \textbf{Средства разработки}{: VSCode, Vim, LaTeX} \\
     \textbf{Средства документации}{: LaTeX}
    }}
 \end{itemize}


%-----------EDUCATION-----------
\section{Образование}
  \resumeSubHeadingListStart
    \resumeSubheading
      {Университет ИТМО}{Санкт-Петербург, РФ}
      {Бакалавриат, Программная инженерия}{Окончание в июле 2026}
          \resumeItemListStart
            \resumeItem{Научное исследование по теме "Архитектура процессора с тегированной памятью"}
            \resumeItem{Разработан эмулятор процессора RISC-V с реализацией кастомного языка ассемблера}
            \resumeItem{Изучение функционирования ядра, внедрение buddy аллокатора, его оптимизация, добавление pthreads, внедрение системы тегирования виртуальной памяти, внедрение ленивой аллокации на базе учебной xv6}
            \resumeItem{Написание кастомной файловой системы}
            \resumeItem{Средний балл: 4.7/5.0}
        \resumeItemListEnd
  \resumeSubHeadingListEnd

% -----------EXPERIENCE-----------
\section{Опыт работы}
  \resumeSubHeadingListStart
    \resumeSubheading
      {Backend-разработчик}{Ноябрь 2023 - Настоящее время}
      {Лаборатория COSM, Команда Crossroads}{Санкт-Петербург, РФ}
      \resumeItemListStart
        \resumeItem{Разработка backend-сервисов с использованием Spring Framework}
        \resumeItem{Проектирование и реализация RESTful API}
      \resumeItemListEnd
    \resumeSubHeadingListEnd

%-----------COMMUNITY AND LEADERSHIP-----------
\section{Сообщество и лидерство}
  \resumeSubHeadingListStart
    \resumeSubheading
      {Full-Stack разработчик. 3-е место}{Сентябрь 2023}
      {Хакатон "На Севере кодить!"}{Мурманск, РФ}
      \resumeItemListStart
        \resumeItem{Разработано два веб-приложений для системы управления транспортными картами}
        \resumeItem{Мобильное приложение на React Native для управления транспортными картами пользователей}
        \resumeItem{Панель администратора с использованием фреймворка Next.js для управления системой}
      \resumeItemListEnd
    \resumeSubHeadingListEnd

%-----------PROJECTS-----------
\section{Проекты}

    \resumeSubHeadingListStart
      \resumeProjectHeading
        {\textbf{Система управления задачами} $|$ \emph{Spring Boot, Angular, PostgreSQL} $|$ \emph{Full-Stack}}{Октябрь 2024}
        \resumeItemListStart
          \resumeItem{Разработка системы управления проектами по типу Jira с планированием спринтов и командным взаимодействием}
          \resumeItem{Реализованы миграции базы данных с помощью Flyway и оптимизирована производительность запросов с помощью индексов}
          \resumeItem{Созданы хранимые процедуры и функции на уровне базы данных для сложной бизнес-логики}
          \resumeItem{Разработан комплексный набор автотестов с использованием Mockito для тестирования API}
        \resumeItemListEnd

      

      \resumeProjectHeading
          {\textbf{Система управления реестром транспорта} $|$ \emph{Spring Boot, React, PostgreSQL, MinIO} $|$ \emph{Full-Stack}}{Сентябрь 2024}
          \resumeItemListStart
            \resumeItem{Система управления реестром транспорта поддерживает распределённые транзакции}
            \resumeItem{Поддерживается возможность импорта скриптов для заполнения системы данными, файлы импорта сохраняются в хранилище MinIO}
            \resumeItem{Для распределённых траназкций вручную реализован двухфазный коммит}
            \resumeItem{Созданы тестовые сценарии в Jmeter, по итогам которых, расставлены пропагейшены и изоляции транзакций. Некоторые транзации настроены полностью вручную и использованием JPA и JTA}
            \resumeItem{Добавлена защищенная панель администратора с использованием Spring Security и JWT}
            \resumeItem{Реализован автоматизированный процесс развертывания с использованием Docker и Nginx}
            \resumeItem{Разработан адаптивный пользовательский интерфейс на React с использованием WebSocket}
      \resumeItemListEnd

      \resumeProjectHeading
          {\textbf{Калькулятор бухгалтерских услуг} $|$ \emph{Spring Boot, React, PostgreSQL} $|$ \emph{Full-Stack}}{Июль 2024}
          \resumeItemListStart
            \resumeItem{Разработана автоматизированная система расчета стоимости бухгалтерских услуг}
            \resumeItem{Реализована панель администратора с аутентификацией JWT и Spring Security}
            \resumeItem{Создана система уведомлений по email с собственным доменом для обратной связи}
        \resumeItemListEnd

    \resumeSubHeadingListEnd



%-------------------------------------------
\end{document}
