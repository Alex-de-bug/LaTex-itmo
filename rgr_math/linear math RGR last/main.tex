\documentclass{article}
\usepackage[utf8]{inputenc} %кодировка
\usepackage[T2A]{fontenc}
\usepackage[english,russian]{babel} %русификатор 
\usepackage{mathtools} %библиотека матеши
\usepackage[left=1cm,right=1cm,top=2cm,bottom=2cm,bindingoffset=0cm]{geometry} %изменение отступов на листе
\usepackage{amsmath}
\usepackage{graphicx} %библиотека для графики и картинок
\graphicspath{}
\DeclareGraphicsExtensions{.pdf,.png,.jpg}
\usepackage{subcaption}
\usepackage{pgfplots}
\usepackage{amsfonts}
\usepackage{tikz}
\usepackage{tikz-3dplot}
\usepackage{caption}
\usepackage{pgfplots}


\usepackage{tikz,tikz-3dplot}

%% style for surfaces
\tikzset{surface/.style={draw=black, left color=yellow,right color=yellow,middle
        color=yellow!60!#1, fill opacity=1},surface/.default=white}

%% macros to draw back and front of cones
%% optional first argument is styling; others are z, radius, side offset (in degrees)
\newcommand{\coneback}[4][]{
    %% start at the correct point on the circle, draw the arc, then draw to the origin of the diagram, then close the path
    \draw[canvas is xy plane at z=#2, #1] (\tdplotmainphi-#4:#3) 
    arc(\tdplotmainphi-#4:\tdplotmainphi+180+#4:#3) -- (O) --cycle;
}
\newcommand{\conefront}[4][]{
    \draw[canvas is xy plane at z=#2, #1] (\tdplotmainphi-#4:#3) arc
    (\tdplotmainphi-#4:\tdplotmainphi-180+#4:#3) -- (O) --cycle;
}

\newcommand{\conetruncback}[7][]{
    \draw[line join=round,#1] plot[variable=\t,domain=\tdplotmainphi-#4:\tdplotmainphi+180+#4] 
    ({#3*cos(\t)},{#3*sin(\t)},#2)
    -- plot[variable=\t,domain=\tdplotmainphi+180-#7:\tdplotmainphi+#7] 
    ({#6*cos(\t)},{#6*sin(\t)},#5)
    --cycle;
}

\newcommand{\conetruncfront}[7][]{
    \draw[line join=round,#1] plot[variable=\t,domain=\tdplotmainphi-#4:\tdplotmainphi-180+#4] 
    ({#3*cos(\t)},{#3*sin(\t)},#2)
    -- plot[variable=\t,domain=\tdplotmainphi-180-#7:\tdplotmainphi+#7] 
    ({#6*cos(\t)},{#6*sin(\t)},#5)
    --cycle;
}




\begin{document}

\thispagestyle{empty}
% НАЧАЛО ТИТУЛЬНОГО ЛИСТА
\begin{center}
        \Large
        Федеральное государственное автономное \\
        образовательное учреждение высшего образования \\ 
        «Научно-образовательная корпорация ИТМО»\\
        \vspace{0.5cm}
        \large
        Факультет программной инженерии и компьютерной техники \\
        Направление подготовки 09.03.04 Программная инженерия \\
        \vspace{1cm}
        \Large
        \textbf{Отчёт по расчетно-графической работе №3} \\
        По дисциплине «Линейная геометрия» (второй семестр)\\
        \large
        \vspace{8cm}
    
        \begin{minipage}{.33\textwidth}
        \end{minipage}
        \hfill
        \begin{minipage}{.4\textwidth}
            \textbf{Группа}: \vspace{.1cm} \\
            \ МАТБАЗ 1.5\\ \\
            \textbf{Студенты}: \vspace{.1cm} \\
            \ Андриянова Софья\\
            \ Беляев Михаил\\
            \ Билошицкий Михаил\\
            \ Дениченко Александр\\
            \ Разинкин Александр\\ \\
            \textbf{Лектор}: \vspace{.1cm} \\
            \ Правдин Константин Владимирович \\ \\
            \textbf{Практик}:  \\
            \ Правдин Константин Владимирович
        \end{minipage}
        \vfill
    Санкт-Петербург\\ 2023 г.
    \end{center}
 
% КОНЕЦ ТИТУЛЬНОГО ЛИСТА

\newpage


% задание 1 начало
\section{Линейный оператор и спектральный анализ}
A)


\textbf{Задание:} Дано пространство геометрических векторов $\mathbb{R}^3$ , его подпространства $L_1$ и $L_2$ и линейный
оператор $\mathcal{A}:\ \mathbb{R}^3 \rightarrow \mathbb{R}^3$. $\mathcal{A}$ – оператор проектирования пространства $\mathbb{R}^3$ на подпространство $L_1$ параллельно
подпространству $L_2$, где $L_1$ определено уравнением x = 0, $L_2$ – уравнениями $2x=2y=-z$\\


% 1) Изобразите на графике подпространства $L_1$ и $L_2$.


% 2) Методами векторной алгебры составьте формулу для линейного оператора


% 3) Составьте его матрицу в базисе $\{i, j, k\}$ пространства $\mathbb{R}^3  $


% 4) Решите задачу о диагонализации полученной матрицы методом спектрального анализа.


% 5) На построенном ранее графике изобразите базис, в котором матрица линейного оператора
% имеет диагональный вид. Объясните его смысл.\\
\textbf{Решение:}\\

1) Изобразим пространства $L_1,\ L_2$:\\

\begin{center}
    \tdplotsetmaincoords{70}{110}

  \begin{tikzpicture}[scale=1.5]
    \tdplotsetrotatedcoords{70}{70}{70}

    % Оси координат
    \draw[thick,->] (0,-3,0) -- (0,3,0) node[anchor=north west]{\(y\)};
    \draw[thick,->] (0,0,-3) -- (0,0,3) node[anchor=south]{\(z\)};
    \draw[thick,->] (0,0,0) -- (3,0,0) node[anchor=north east]{\(x\)};

    % Прямая 2x = 2y = -z (пунктирная)
    
    \draw[red,thick,dashed] (0, 0, 0) -- (-1.4,-1,2) node[anchor=west]{\(L_2\)};
    % Плоскость x = 0
    \fill[blue!20,opacity=0.5] (0,-3,-3) -- (0,-3,3) -- (0,3,3) -- (0,3,-3) -- cycle;
    \node[blue] at (0,2,2) {\(L_1\)};
    \draw[red,thick,] (1.4,1,-2) -- (0, 0 ,0) node[anchor=west]{\(L_2\)};
    % Точка пересечения прямой и плоскости
    \draw[fill=green] (0,0,0) circle (0.05) node[anchor=south east]{\(R\)};
  \end{tikzpicture}
  \captionof{figure}{Пространства}
\end{center}

2) Методами векторной алгебры составим формулу для линейного оператора:


Покажем графически:
\begin{center}
    \tdplotsetmaincoords{70}{110}

  \begin{tikzpicture}[scale=1.3]
    \tdplotsetrotatedcoords{70}{70}{70}
    % Прямая 2x = 2y = -z (пунктирная)
    
    \draw[red,thick,dashed] (0, 0, 0) -- (-1.4,-1,2) node[anchor=west]{\(L_2\)};
    \draw[blue,thick,dashed] (0, 2, 0) -- (-1.4,1,2) node[anchor=west]{\(\)};
    % Плоскость x = 0
    \fill[blue!20,opacity=0.5] (0,-3,-7) -- (0,-3,3) -- (0,3,3) -- (0,3,-7) -- cycle;
    \node[blue] at (0,-2,2) {\(L_1\)};
    \draw[red,thick,] (1.4,1,-2) -- (0, 0 ,0) node[anchor=west]{\(\)};
    \draw[blue,thick, ->] (0, 2 ,0) -- (1.4,3,-2) node[anchor=west]{\(s\)};
    \draw[red,thick, ->] (0, 2 ,0) -- (0.9,2.632,-1.5) node[midway, above left]{\(\overrightarrow{x_1}\)};
    \draw[black,thick, ->] (0, 0 ,0) -- (0.9,2.632,-1.5) node[midway, above right]{\(\overrightarrow{x}\)};
    % Точка пересечения прямой и плоскости
    \draw[fill=green] (0,0,0) circle (0.05) node[anchor=south east]{\(R\)};
    \draw[orange,thick, ->] (0, 0 ,0) -- (1, 0 ,0) node[anchor=west]{\(\overrightarrow{n}\)};
    \draw[blue,thick, ->] (0, 0 ,0) -- (0, 2 ,0) node[midway, above left]{\(\overrightarrow{x_2}\)};
    \end{tikzpicture}
    \captionof{figure}{Вспомогательные построения}
    \end{center}


    Получим формулу для линейного оператора.
    \begin{equation*}
        \overline{x} = \overline{x_1} + \overline{x_2}
    \end{equation*}
    \begin{equation*}
        \overline{x_2} = \overline{x} - \overline{x_1} = \overline{x} - \frac{<\overline{x},\overline{s}>}{||\overline{s}||^2}\cdot \overline{s}
    \end{equation*}   
    
    3) Составим матрицу оператора в базисе $\{i,j,k\}$ в $\mathbb{R}^3$\\
    Пусть $\overline{s}$ имеет координаты $(1; 1; -2)^T$, исходя из данного по условию уравнения $2x=2y=-z$. Возьмём точку $A(1; 0; 0)$ и скажем, что новый вектор проходит через эту точку\\
    При помощи канонического уравнения прямой мы можем получить следующее:
    \begin{equation*}
        \frac{x-x_0}{m}=\frac{y-y_0}{n}=\frac{z-z_0}{p}
    \end{equation*}
    \begin{equation*}
        \frac{x-1}{1}=\frac{y-0}{1}=\frac{z-0}{-2},\ x-1=y=\frac{z}{-2},\ -2x+2=-2y=z
    \end{equation*}
    \begin{equation*}
        \begin{cases}
            x=0\\
            -2x+2=-2y\\
            -2y=z\\
            -2x+2=z
        \end{cases}
        \begin{cases}
            x=0\\
            y=-1\\
            z=2
        \end{cases}
    \end{equation*}
Мы получили действие оператором на базисный вектор i.\\
    \begin{equation*}
        \mathcal{A} i = 
    \begin{pmatrix}
        0 \\
        -1\\
        2
    \end{pmatrix}, 
    \mathcal{A} j = 
    \begin{pmatrix}
        0 \\
        1\\
        0
    \end{pmatrix}
    , 
    \mathcal{A} k = 
    \begin{pmatrix}
        0 \\
        0\\
        1
    \end{pmatrix}
    \end{equation*}
    Матрица оператора будет иметь вид:\\
    \begin{equation*}
    A = 
    \begin{pmatrix}
        0&0&0 \\
        -1&1&0\\
        2&0&1
    \end{pmatrix}
    \end{equation*}


    4) Диагонализуем полученную матрицу.
    \begin{equation*}
        det(A-\lambda I)
    \end{equation*}
    \begin{equation*}
        \begin{pmatrix}
            -\lambda&0&0 \\
            -1&1-\lambda&0\\
            2&0&1-\lambda
        \end{pmatrix}=-\lambda(\lambda-1)^2
        \end{equation*}
    $\lambda_1=0$(кр 1):\\
    \begin{equation*}
        \begin{pmatrix}
            0&0&0 \\
            -1&1&0\\
            2&0&1
        \end{pmatrix}\sim 
        \begin{pmatrix}
            1&0|&1 \\
            0&1|&-2
        \end{pmatrix},\ x_1=C_1
        \begin{pmatrix}
            1\\
            1\\
            -2
        \end{pmatrix}
    \end{equation*}
    $\lambda_2=1$(кр 2):\\
    \begin{equation*}
        \begin{pmatrix}
            -1&0&0 \\
            -1&0&0\\
            2&0&0
        \end{pmatrix}\sim 
        \begin{pmatrix}
            1&0&0 
        \end{pmatrix}\sim 
        \begin{pmatrix}
            1&0|&0 
        \end{pmatrix},\ x_2=C_2
        \begin{pmatrix}
            1\\
            1\\
            0
        \end{pmatrix}+
        C_3\begin{pmatrix}
            1\\
            0\\
            1
        \end{pmatrix}
    \end{equation*}
Итоговая диагонализованная матрица:\\
\begin{equation*}A^*=
    \begin{pmatrix}
        0&0&0 \\
        0&1&0\\
        0&0&1
    \end{pmatrix}
\end{equation*}
5) Базис в котором матрица оператора имеет диагонаольный вид:\\
\begin{center}
    \tdplotsetmaincoords{70}{100}

  \begin{tikzpicture}[scale=1]
    \tdplotsetrotatedcoords{90}{70}{70}
    \draw[thick,dashed,->] (0,0,0) -- (2,1,-2) node[anchor=north west]{\(\)};
    \fill[blue!20,opacity=0.5] (0,0,0) -- (3,0,0) -- (3,3,0) -- (0,3,0) -- cycle;
    % Оси координат
    \draw[thick,->] (0,0,0) -- (0,3,0) node[anchor=north west]{\(y\)};
    \draw[thick,->] (0,0,0) -- (0,0,3) node[anchor=south]{\(z\)};
    \draw[thick,->] (0,0,0) -- (3,0,0) node[anchor=north east]{\(x\)};

    % Оси координат
    
    \draw[thick,->] (0,0,0) -- (1,1,0) node[anchor=south]{\(\)};
    \draw[thick,->] (0,0,0) -- (1,0,1) node[anchor=north east]{\(\)};

    
    
  \end{tikzpicture}
  \captionof{figure}{Базис}
\end{center}

Б)


\textbf{Задание:} Дано множество функций $L$ и отображение $\mathcal{A}: L \rightarrow L$. $L$ - множество функций вида \\$y=a\ cos2t+b\ sin2t+c\ tcos2t+d\ tsin2t$, где $a,b,c,d \in \mathbb{R},\ \mathcal{A} =\mathcal{D}^2+4\mathcal{I}$, $\mathcal{D} $- дифференцирование, т.е. $\mathcal{D}(y(t))=\frac{dy}{dt}. $\\


% 1) Изобразите на графике подпространства $L_1$ и $L_2$.


% 2) Методами векторной алгебры составьте формулу для линейного оператора


% 3) Составьте его матрицу в базисе $\{i, j, k\}$ пространства $\mathbb{R}^3  $


% 4) Решите задачу о диагонализации полученной матрицы методом спектрального анализа.


% 5) На построенном ранее графике изобразите базис, в котором матрица линейного оператора
% имеет диагональный вид. Объясните его смысл.\\
\textbf{Решение:}\\


1) Проверим, что множество функций данное по условию является линейным пространством над полем $\mathbb{R} $


Давайте проверим каждую из аксиом для данного множества функций:
\\
1. Закон сложения:\\
   Закон коммутативности: Проверим \(f(t) + g(t) = g(t) + f(t)\):
   \begin{equation*}
    f(t) + g(t) = (a \cos(2t) + b \sin(2t) + c t \cos(2t) + d t \sin(2t))
     + (a' \cos(2t) + b' \sin(2t) + c' t \cos(2t) + d' t \sin(2t))
     =
   \end{equation*}
   \begin{equation*}
    = (a + a') \cos(2t) + (b + b') \sin(2t) 
   + (c + c') t \cos(2t) + (d + d') t \sin(2t)
   \end{equation*}
     Проверка показывает, что \(f(t) + g(t) = g(t) + f(t)\), что соответствует закону коммутативности.\\
   \\Закон ассоциативности: Проверим \((f(t) + g(t)) + h(t) = f(t) + (g(t) + h(t))\):
   \begin{equation*}
    (f(t) + g(t)) + h(t) = ((a \cos(2t) + b \sin(2t) + c t \cos(2t) + d t \sin(2t))
     + (a' \cos(2t) + b' \sin(2t) + c' t \cos(2t) + d' t \sin(2t))) 
     + (a'' \cos(2t) + 
   \end{equation*}
   \begin{equation*}
    + b'' \sin(2t) + c'' t \cos(2t) + d'' t \sin(2t))= (a + a' + a'') \cos(2t) + (b + b' + b'') \sin(2t)
     + (c + c' + c'') t \cos(2t) + (d + d' + d'') t \sin(2t)
   \end{equation*}
     и
     \begin{equation*}
        f(t) + (g(t) + h(t)) = (a \cos(2t) + b \sin(2t) + c t \cos(2t) + d t \sin(2t))
      + ((a' \cos(2t) + b' \sin(2t) + c' t \cos(2t) + d' t \sin(2t)) 
      + (a'' \cos(2t) + b'' \sin(2t) +
       \end{equation*}
       \begin{equation*}
        + c'' t \cos(2t) + d'' t \sin(2t))) 
     = (a + a' + a'') \cos(2t) + (b + b' + b'') \sin(2t) 
      + (c + c' + c'') t \cos(2t) + (d + d' + d'') t \sin(2t)
       \end{equation*}
       Проверка показывает, что \((f(t) + g(t)) + h(t) = f(t) + (g(t) + h(t))\), что соответствует закону ассоциативности.
       \\ \\
       Существование нулевого элемента: Найдем значения коэффициентов \(a, b, c, d\) такие, чтобы функция \(0(t) = 0\) для любого \(t\). Это означает, что все коэффициенты должны быть равны нулю.
       Если мы положим \(a = b = c = d = 0\), то получим \(y = 0\) для всех \(t\), что соответствует нулевой функции.
       \\ \\
     Существование противоположного элемента: Для каждого коэффициента \(a, b, c, d\) найдем противоположные значения, так чтобы сумма функций равнялась нулевой функции.
       Например, если у нас есть функция \(y = a \cos(2t) + b \sin(2t) + c t \cos(2t) + d t \sin(2t)\), то противоположная функция будет иметь коэффициенты \(-a, -b, -c, -d\) и будет выглядеть как \(-a \cos(2t) - b \sin(2t) - c t \cos(2t) - d t \sin(2t)\).
\\ \\
  2. Закон умножения на скаляр:\\
\begin{itemize}
  \item Закон ассоциативности: $\forall \alpha, \beta \in \mathbb{R}$, $\forall f(t) \in \mathcal{F}$
  \[
  \alpha (\beta f(t)) = (\alpha \beta) f(t)
  \]
  
  \item Закон дистрибутивности по скаляру: $\forall \alpha, \beta \in \mathbb{R}$, $\forall f(t), g(t) \in \mathcal{F}$
  \[
  (\alpha + \beta) f(t) = \alpha f(t) + \beta f(t)
  \]
\end{itemize}

\textbf{Закон ассоциативности:}

Пусть $\alpha, \beta \in \mathbb{R}$ и $f(t) = a\cos(2t) + b\sin(2t) + ct\cos(2t) + dt\sin(2t)$.

Тогда $\alpha (\beta f(t))$ будет равно:
\[
\alpha (\beta f(t)) = \alpha \left(\beta (a\cos(2t) + b\sin(2t) + ct\cos(2t) + dt\sin(2t))\right)
\]
\[
= (\alpha \beta) (a\cos(2t) + b\sin(2t) + ct\cos(2t) + dt\sin(2t))
\]
\[
= (\alpha \beta) f(t)
\]

Таким образом, закон ассоциативности выполняется.

\textbf{Закон дистрибутивности по скаляру:}

Пусть $\alpha, \beta \in \mathbb{R}$ и $f(t), g(t) \in \mathcal{F}$, где $f(t) = a\cos(2t) + b\sin(2t) + ct\cos(2t) + dt\sin(2t)$ и $g(t) = p\cos(2t) + q\sin(2t) + rt\cos(2t) + st\sin(2t)$.

Тогда $(\alpha + \beta) f(t)$ будет равно:
\[
(\alpha + \beta) f(t) = (\alpha + \beta)(a\cos(2t) + b\sin(2t) + ct\cos(2t) + dt\sin(2t))
\]
\[
= \alpha (a\cos(2t) + b\sin(2t) + ct\cos(2t) + dt\sin(2t)) + \beta (a\cos(2t) + b\sin(2t) + ct\cos(2t) + dt\sin(2t))
\]
\[
= \alpha f(t) + \beta f(t)
\]

Таким образом, мы проверили, что Закон умножения на скаляр выполняется для множества функций $y=a\cos(2t)+b\sin(2t)+ct\cos(2t)+dt\sin(2t)$, и можем сделать вывод, что данное множество является линейным пространством над полем вещественных чисел.
\\ \\
\\ \\
2) Выберем базис пространства.\\
\begin{equation*}
    \{cos(2t); sin(2t); t\cdot cos(2t);t\cdot sin(2t)\}
\end{equation*}
Получим матрицу оператора $\mathcal{A}$. Для начала исследуем действие оператора $\mathcal{D} $ на наш базис.\\
\begin{equation*}
    (cos(2t))'=-2sin(2t)
\end{equation*}
\begin{equation*}
    (sin(2t))'=2cos(2t)
\end{equation*}
\begin{equation*}
    (tcos(2t))'=cos(2t) - 2tsin(2t)
\end{equation*}
\begin{equation*}
    (tsin(2t))'=sin(2t) + 2tcos(2t)
\end{equation*}
Матрица $\mathcal{D} $ будет выглядеть следующим образом:\
\begin{equation*}
    \begin{pmatrix}
        0&2&1&0\\
        -2&0&0&1\\
        0&0&0&2\\
        0&0&-2&0
    \end{pmatrix}
\end{equation*}
Матрица $\mathcal{A} $ тогда будет выглядеть:
\begin{equation*}
    \begin{pmatrix}
        0&2&1&0\\
        -2&0&0&1\\
        0&0&0&2\\
        0&0&-2&0
    \end{pmatrix}^2-
    \begin{pmatrix}
        4&0&0&0\\
        0&4&0&0\\
        0&0&4&0\\
        0&0&0&4
    \end{pmatrix}=
    \begin{pmatrix}
        0&0&0&4\\
        0&0&-4&0\\
        0&0&0&0\\
        0&0&0&0
    \end{pmatrix}
\end{equation*}
3) Проверим линейность оператора.\\


Для проверки линейности оператора, представленного матрицей, мы должны проверить два условия:

1. Сохранение сложения:
   Для любых функций $f(t)$ и $g(t)$ должно выполняться $T(f(t) + g(t)) = T(f(t)) + T(g(t))$.

2. Сохранение умножения на скаляр:
   Для любой функции $f(t)$ и любого скаляра $\alpha$ должно выполняться $T(\alpha f(t)) = \alpha T(f(t))$.

Проверим оба этих условия для данного оператора с матрицей:
\[
\begin{pmatrix}
0 & 0 & 0 & 4 \\
0 & 0 & -4 & 0 \\
0 & 0 & 0 & 0 \\
0 & 0 & 0 & 0 \\
\end{pmatrix}
\]

1. Сохранение сложения:
   Пусть $f(t)$ и $g(t)$ являются произвольными функциями. Тогда
   \begin{align*}
   T(f(t) + g(t)) &= \begin{pmatrix} 0 & 0 & 0 & 4 \\ 0 & 0 & -4 & 0 \\ 0 & 0 & 0 & 0 \\ 0 & 0 & 0 & 0 \end{pmatrix} \begin{pmatrix} (a+a_1)\cos(2t) + (b+b_1)\sin(2t) + (c+c_1)t\cos(2t) + (d+d_1)t\sin(2t) \\
   (e+e_1)\cos(2t) + (f+f_1)\sin(2t) + (g+g_1)t\cos(2t) + (h+h_1)t\sin(2t) \\ 
   (j+j_1)\cos(2t) + (k+k_1)\sin(2t) + (l+l_1)t\cos(2t) + (m+m_1)t\sin(2t) \\ 
   (n+n_1)\cos(2t) + (p+p_1)\sin(2t) + (q+q_1)t\cos(2t) + (r+r_1)t\sin(2t) \end{pmatrix} \\
   &= \begin{pmatrix} 4(n+n_1)\cos(2t) + 4(p+p_1)\sin(2t) + 4(q+q_1)t\cos(2t) + 4(r+r_1)t\sin(2t) \\ 
    -4(j+j_1)\cos(2t) -4(k+k_1)\sin(2t) -4(l+l_1)t\cos(2t) -4(m+m_1)t\sin(2t) \\ 0 \\ 0 \end{pmatrix} \\
   &= \begin{pmatrix} 4(n)\cos(2t) + 4(p)\sin(2t) + 4(q)t\cos(2t) + 4(r)t\sin(2t) \\ 
    -4(j)\cos(2t) -4(k)\sin(2t) -4(l)t\cos(2t) -4(m)t\sin(2t) \\ 0 \\ 0 \end{pmatrix} +\\
    &+ \begin{pmatrix} 4(n_1)\cos(2t) + 4(p_1)\sin(2t) + 4(q_1)t\cos(2t) + 4(r_1)t\sin(2t) \\ 
        -4(j_1)\cos(2t) -4(k_1)\sin(2t) -4(l_1)t\cos(2t) -4(m_1)t\sin(2t) \\ 0 \\ 0 \end{pmatrix}= \\
   &= T(f(t)) + T(g(t))
   \end{align*}
   Таким образом, условие сохранения сложения выполняется.

2. Сохранение умножения на скаляр:
   Пусть $f(t)$ является произвольной функцией, и $\alpha$ является произвольным скаляром. Тогда
   \begin{align*}
   T(\alpha f(t)) &= \begin{pmatrix} 0 & 0 & 0 & 4 \\ 0 & 0 & -4 & 0 \\ 0 & 0 & 0 & 0 \\ 0 & 0 & 0 & 0 \end{pmatrix} 
   \begin{pmatrix}\alpha(a\cos(2t) + b\sin(2t) + ct\cos(2t) + dt\sin(2t)) \\ \alpha(e\cos(2t) + f\sin(2t) + gt\cos(2t) + ht\sin(2t)) \\ \alpha(j\cos(2t) + k\sin(2t) + lt\cos(2t) + mt\sin(2t)) \\ \alpha(n\cos(2t) + p\sin(2t) + qt\cos(2t) + rt\sin(2t)) \end{pmatrix} \\
   &= \begin{pmatrix}\alpha4(n\cos(2t) + p\sin(2t) + qt\cos(2t) + rt\sin(2t)) \\ -\alpha4(j\cos(2t) + k\sin(2t) + lt\cos(2t) + mt\sin(2t)) \\ 0 \\ 0 \end{pmatrix} \\
   &= \alpha \begin{pmatrix}4(n\cos(2t) + p\sin(2t) + qt\cos(2t) + rt\sin(2t)) \\ -4(j\cos(2t) + k\sin(2t) + lt\cos(2t) + mt\sin(2t)) \\ 0 \\ 0 \end{pmatrix} \\
   &= \alpha T(f(t))
   \end{align*}
   Таким образом, условие сохранения умножения на скаляр также выполняется.

Исходя из результатов проверки обоих условий, мы можем заключить, что данный оператор является линейным оператором.
\\ \\
\\ \\
4) Найдём размерности ядра и образа оператора.\\
\begin{equation*}
    \begin{pmatrix}
        0 & 0 & 0 & 4 \\
        0 & 0 & -4 & 0 \\
        0 & 0 & 0 & 0 \\
        0 & 0 & 0 & 0 \\
        \end{pmatrix}\sim 
        \begin{pmatrix}
            0 & 0 & 0 & 1 \\
            0 & 0 & 1 & 0 \\
            \end{pmatrix}
\end{equation*}
\begin{equation*}
    dim\ Ker(L) = 2;\ dim\ Im(L)= dim(L)- dim\ Ker(L) = 4 -2=2
\end{equation*}
\\ \\
\\ \\
5) Диагонализуем матрицу оператора.\\
\begin{equation*}
    \begin{pmatrix}
        -\lambda & 0 & 0 & 4 \\
        0 & -\lambda & -4 & 0 \\
        0 & 0 & -\lambda & 0 \\
        0 & 0 & 0 & -\lambda \\
        \end{pmatrix}= -\lambda^4
\end{equation*}
$\lambda=0 $(кр. = 4):
\begin{equation*}
    \begin{pmatrix}
        0 & 0 & 0 & 4 \\
        0 & 0 & -4 & 0 \\
        0 & 0 & 0 & 0 \\
        0 & 0 & 0 & 0 \\
        \end{pmatrix}\sim 
        \begin{pmatrix}
            0 & 0 & 0 & 1 \\
            0 & 0 & 1 & 0 \\
        \end{pmatrix}\sim 
        \begin{pmatrix}
            0 & 1| & 0 & 0 \\
            1 & 0| & 0 & 0 \\
        \end{pmatrix}
\end{equation*}
\begin{equation*}
    X_1=C_1\cdot 
    \begin{pmatrix}
        1 \\
        0 \\
        0 \\
        0 \\
    \end{pmatrix}+
    C_2
    \cdot
    \begin{pmatrix}
        0 \\
        1 \\
        0 \\
        0 \\
    \end{pmatrix}
\end{equation*}
Нам нужно присоединить 2 вектора, так как кратности не совпадают.\\
Присоединим первый вектор к составляющей $X_1$ с коэфицентом $C_1$:
\begin{equation*}
    \begin{pmatrix}
        0 & 0 & 0 & 4|&1 \\
        0 & 0 & -4 & 0|&0 \\
        0 & 0 & 0 & 0|&0 \\
        0 & 0 & 0 & 0|&0 \\
        \end{pmatrix}\sim 
        \begin{pmatrix}
            0 & 0 & 0 & 1|&\frac{1}{4} \\
            0 & 0 & 1 & 0|&0 \\
        \end{pmatrix}
\end{equation*}
\begin{equation*}
    X^{(1)}_1=C_1\cdot 
    \begin{pmatrix}
        0 \\
        0 \\
        0 \\
        \frac{1}{4} \\
    \end{pmatrix}+
    C^{(1)}_2\cdot 
    \begin{pmatrix}
        1 \\
        0 \\
        0 \\
        0 \\
    \end{pmatrix}+
    C^{(2)}_2\cdot 
    \begin{pmatrix}
        0 \\
        1 \\
        0 \\
        0 \\
    \end{pmatrix}
\end{equation*}
Нам нужно присоединить второй вектор, так как кратности попрежнему не совпадают.\\
Присоединим второй вектор к составляющей $X_1$ с коэфицентом $C_2$:
\begin{equation*}
    \begin{pmatrix}
        0 & 0 & 0 & 4|&1 \\
        0 & 0 & -4 & 0|&0 \\
        0 & 0 & 0 & 0|&0 \\
        0 & 0 & 0 & 0|&0 \\
        \end{pmatrix}\sim 
        \begin{pmatrix}
            0 & 0 & 0 & 1|& 0\\
            0 & 0 & 1 & 0|&-\frac{1}{4} \\
        \end{pmatrix}
\end{equation*}
\begin{equation*}
    X^{(2)}_1=C_2\cdot 
    \begin{pmatrix}
        0 \\
        0 \\
        -\frac{1}{4} \\
        0 \\
    \end{pmatrix}+
    C^{(3)}_2\cdot 
    \begin{pmatrix}
        1 \\
        0 \\
        0 \\
        0 \\
    \end{pmatrix}+
    C^{(4)}_2\cdot 
    \begin{pmatrix}
        0 \\
        1 \\
        0 \\
        0 \\
    \end{pmatrix}
\end{equation*}
Кратности совпали, теперь можно записать диагонализованную матрицу в Жордановом базисе:\\
\begin{equation*}\mathcal{A}^{\mathcal{J} } =
    \begin{pmatrix}
        0 & 1 & 0 & 0 \\
        0 & 0 & 0 & 0 \\
        0 & 0 & 0 & 1 \\
        0 & 0 & 0 & 0 \\
        \end{pmatrix}
\end{equation*}
6) Выберем произвольно и нетривиально функцию. Найдём её образ умножением на матрицу оператора.
\begin{equation*}
   \vartheta (x)= 1\cdot \cos(2t) + 2\cdot \sin(2t) + 3\cdot t \cos(2t) + 4\cdot t \sin(2t)
\end{equation*}
\begin{equation*}
    \begin{pmatrix}
        0 & 0 & 0 & 4 \\
        0 & 0 & -4 & 0 \\
        0 & 0 & 0 & 0 \\
        0 & 0 & 0 & 0 \\
        \end{pmatrix}\cdot 
        \begin{pmatrix}
            1 \\
            2 \\
            3 \\
            4 \\
            \end{pmatrix}=
            \begin{pmatrix}
                16 \\
                -12 \\
                0 \\
                0 \\
                \end{pmatrix}
    \end{equation*}
    Действием оператора получили следующую функцию:
    \begin{equation*}
        \psi (x)= 16\cdot \cos(2t) - 12\cdot \sin(2t)
     \end{equation*}
     Найдём непосредственными вычислениями образ:
     \begin{equation*}
        \mathcal{D}(\vartheta): \vartheta' (x)= -6\,t\,\sin\left(2\,t\right)+2\,\sin\left(2\,t\right)+8\,t\,\cos\left(2\,t\right)+7\,\cos\left(2\,t\right)
     \end{equation*}
     \begin{equation*}
        \mathcal{D}(\vartheta'): \vartheta'' (x)= -16\,t\,\sin\left(2\,t\right)-20\,\sin\left(2\,t\right)-12\,t\,\cos\left(2\,t\right)+12\,\cos\left(2\,t\right)
     \end{equation*}
     \begin{equation*}
        4\mathcal{I}: 4\cdot \vartheta (x)= 4\cdot \cos(2t) + 8\cdot \sin(2t) + 12\cdot t \cos(2t) + 16\cdot t \sin(2t)
     \end{equation*}
     \begin{equation*}
        \zeta  (x)= \vartheta'' (x) - 4\cdot \vartheta (x) = -16\,t\,\sin\left(2\,t\right)-20\,\sin\left(2\,t\right)-12\,t\,\cos\left(2\,t\right)+12\,\cos\left(2\,t\right)+4\cdot \cos(2t) + 8\cdot \sin(2t) + 12\cdot t \cos(2t) + 16\cdot t \sin(2t)=
     \end{equation*}
     \begin{equation*}
        = 16\cdot \cos(2t) - 12\cdot \sin(2t) = \psi (x)
     \end{equation*}
     Вычисления сошлись, удобнее было вычислять через матрицу оператора.

\section{Евклидовы пространства функций}

     A)


\textbf{Задание:} Дано пространство многочленов с вещественными коэффициентами, степени не выше третьей,
определенных на отрезке [– 1; 1]. $P_3(t) = t^3+2t^2-2t+1$\\

\textbf{Решение:}\\


1)Проверим, что система векторов  $B= {1, t, t^2,t^3}$ является базисом этого пространства.\\
Составим матрицу Грама и посчитаем её опередлитель:\\

\begin{equation*}
    det\left(
        \begin{pmatrix}
            \int_{-1}^{1}1dt & \int_{-1}^{1}tdt & \int_{-1}^{1}t^2dt & \int_{-1}^{1}t^3dt \\
            \int_{-1}^{1}tdt & \int_{-1}^{1}t^2dt & \int_{-1}^{1}t^3dt & \int_{-1}^{1}t^4dt \\
            \int_{-1}^{1}t^2dt & \int_{-1}^{1}t^3dt & \int_{-1}^{1}t^4dt & \int_{-1}^{1}t^5dt \\
            \int_{-1}^{1}t^3dt & \int_{-1}^{1}t^4dt & \int_{-1}^{1}t^5dt & \int_{-1}^{1}t^6dt \\
        \end{pmatrix}\right)
 \end{equation*}
 Для вычисления определителя матрицы, можно использовать различные методы, например, метод разложения по определённой строке или столбцу, метод Гаусса и т.д. В данном случае, определитель может быть вычислен с помощью разложения по первому столбцу:

\[
\begin{vmatrix}
    \int_{-1}^{1}1dt & \int_{-1}^{1}tdt & \int_{-1}^{1}t^2dt & \int_{-1}^{1}t^3dt \\
    \int_{-1}^{1}tdt & \int_{-1}^{1}t^2dt & \int_{-1}^{1}t^3dt & \int_{-1}^{1}t^4dt \\
    \int_{-1}^{1}t^2dt & \int_{-1}^{1}t^3dt & \int_{-1}^{1}t^4dt & \int_{-1}^{1}t^5dt \\
    \int_{-1}^{1}t^3dt & \int_{-1}^{1}t^4dt & \int_{-1}^{1}t^5dt & \int_{-1}^{1}t^6dt \\
\end{vmatrix}
= 
\begin{vmatrix}
    2 & 0 & \frac{2}{3} & 0 \\
    0 & \frac{2}{3} & 0 &\frac{2}{5} \\
    \frac{2}{3} & 0 & \frac{2}{5} & 0 \\
    0 & \frac{2}{5} & 0 & \frac{2}{7} \\
\end{vmatrix}= \frac{256}{23625}
\]


Если определитель матрицы Грама не равен нулю, то это означает, что векторы образуют базис.\\

Для ортогонализации системы функций ${1, t, t^2,t^3}$ на отрезке $[-1,1]$ методом Грама-Шмидта, мы последовательно вычислим ортогональные функции:

1: Первая ортогональная функция \(f_1(t)\) остается равной исходной функции 1.

2: Вторая ортогональная функция \(f_2(t)\) будет равна разности второй функции и проекции второй функции на первую функцию:
\[
f_2(t) = t - \frac{\langle t, f_1 \rangle}{\langle f_1, f_1 \rangle} \cdot f_1(t) = t - \frac{\int_{-1}^{1} t \cdot 1 \, dt}{\int_{-1}^{1} 1 \cdot 1 \, dt} \cdot 1 = t - \frac{0}{2} = t
\]

3: Третья ортогональная функция \(f_3(t)\) будет равна разности третьей функции и проекции третьей функции на первую и вторую функции:
\[
f_3(t) = t^2 - \frac{\langle t^2, f_1 \rangle}{\langle f_1, f_1 \rangle} \cdot f_1(t) - \frac{\langle t^2, f_2 \rangle}{\langle f_2, f_2 \rangle} \cdot f_2(t) = t^2 - \frac{\int_{-1}^{1} t^2 \cdot 1 \, dt}{\int_{-1}^{1} 1 \cdot 1 \, dt} \cdot 1 - \frac{\int_{-1}^{1} t^2 \cdot t \, dt}{\int_{-1}^{1} t \cdot t \, dt} \cdot t = t^2 - \frac{2}{6}
\]

4: Четвертая ортогональная функция \(f_4(t)\) будет равна разности четвертой функции и проекции четвертой функции на первую, вторую и третью функции:
\[
f_4(t) = t^3 - \frac{\langle t^3, f_1 \rangle}{\langle f_1, f_1 \rangle} \cdot f_1(t) - \frac{\langle t^3, f_2 \rangle}{\langle f_2, f_2 \rangle} \cdot f_2(t) - \frac{\langle t^3, f_3 \rangle}{\langle f_3, f_3 \rangle} \cdot f_3(t) = t^3 - \frac{\int_{-1}^{1} t^3 \cdot 1 \, dt}{\int_{-1}^{1} 1 \cdot 1 \, dt} \cdot 1 - \frac{\int_{-1}^{1} t^3 \cdot t \, dt}{\int_{-1}^{1} t \cdot t \, dt} \cdot t - \frac{\int_{-1}^{1} t^3 \cdot (t^2 - \frac{2}{6}) \, dt}{\int_{-1}^{1} (t^2 - \frac{2}{6}) \cdot (t^2 - \frac{2}{6}) \, dt} \cdot (t^2 - \frac{2}{6})=t^3-\frac{2}{5}t
\]

Таким образом, получаем ортогональную систему функций {\(f_1(t)\), \(f_2(t)\), \(f_3(t)\), \(f_4(t)\)} на отрезке \([-1, 1]\)\\



2)Выпишем первые четыре многочлена Лежандра:\\
Многочлены Лежандра \(P_n(x)\) ещё определяются рекурсивной формулой:

\[
P_0(x) = 1
\]
\[
P_1(x) = x
\]
\[
P_n(x) = \frac{{(2n-1)xP_{n-1}(x) - (n-1)P_{n-2}(x)}}{{n}}
\]

Таким образом, первые четыре многочлена Лежандра равны: \(P_0(x) = 1\), \(P_1(x) = x\), \(P_2(x) = \frac{{3x^2 - 1}}{{2}}\), \(P_3(x) = \frac{{5x^3 - 3x}}{{2}}\).\\


3) Чтобы найти координаты многочленов в заданном базисе, нужно разложить каждый многочлен по базисным функциям и выразить их коэффициенты. В данном случае базис состоит из ${1, t, t^2-\frac{2}{6}, t^3-\frac{2}{5}t}$.
Для многочлена \(P_0(x) = 1\):
\[P_0(x) = 1 \cdot 1 + 0 \cdot t + 0 \cdot \left(t^2-\frac{2}{6}\right) + 0 \cdot \left(t^3-\frac{2}{5}t\right)\]
Координаты многочлена \(P_0(x)\) в данном базисе равны (1, 0, 0, 0).

Для многочлена \(P_1(x) = x\):
\[P_1(x) = 0 \cdot 1 + 1 \cdot t + 0 \cdot \left(t^2-\frac{2}{6}\right) + 0 \cdot \left(t^3-\frac{2}{5}t\right)\]
Координаты многочлена \(P_1(x)\) в данном базисе равны (0, 1, 0, 0).

Для многочлена \(P_2(x) = \frac{{3x^2 - 1}}{{2}}\):
\[P_2(x) = 0 \cdot 1 + 0 \cdot t + \frac{3}{2} \cdot \left(t^2-\frac{2}{6}\right) + 0 \cdot \left(t^3-\frac{2}{5}t\right)\]
Координаты многочлена \(P_2(x)\) в данном базисе равны (0, 0, \(\frac{3}{2}\), 0).

Для многочлена \(P_3(x) = \frac{{5x^3 - 3x}}{{2}}\):
\[P_3(x) = 0 \cdot 1 + 0 \cdot t + 0 \cdot \left(t^2-\frac{2}{6}\right) + \frac{5}{2} \cdot \left(t^3-\frac{2}{5}t\right)\]
Координаты многочлена \(P_3(x)\) в данном базисе равны (0, 0, 0, \(\frac{5}{2}\)).

Таким образом, координаты многочленов \(P_0(x)\), \(P_1(x)\), \(P_2(x)\), \(P_3(x)\) в базисе ${1, t, t^2-\frac{2}{6}, t^3-\frac{2}{5}t}$ соответственно равны 
\begin{center}
    $(1, 0, 0, 0),\ (0, 1, 0, 0),\ (0, 0, \frac{3}{2}, 0),\ (0, 0, 0, \frac{5}{2})$.
\end{center}


Для проверки ортогональности системы векторов в новом базисе, мы должны вычислить их скалярные произведения и убедиться, что они равны нулю.

Пусть система векторов в новом базисе задана как \(\{v_0, v_1, v_2, v_3\}\) соответственно, где
\(v_0 = (1, 0, 0, 0)\),
\(v_1 = (0, 1, 0, 0)\),
\(v_2 = (0, 0, \frac{3}{2}, 0)\),
\(v_3 = (0, 0, 0, \frac{5}{2})\).

Вычислим скалярные произведения между парами векторов:


\begin{align*}
v_0 \cdot v_1 &= (1, 0, 0, 0) \cdot (0, 1, 0, 0) = 0 \\
v_0 \cdot v_2 &= (1, 0, 0, 0) \cdot (0, 0, \frac{3}{2}, 0) = 0 \\
v_0 \cdot v_3 &= (1, 0, 0, 0) \cdot (0, 0, 0, \frac{5}{2}) = 0 \\
v_1 \cdot v_2 &= (0, 1, 0, 0) \cdot (0, 0, \frac{3}{2}, 0) = 0 \\
v_1 \cdot v_3 &= (0, 1, 0, 0) \cdot (0, 0, 0, \frac{5}{2}) = 0 \\
v_2 \cdot v_3 &= (0, 0, \frac{3}{2}, 0) \cdot (0, 0, 0, \frac{5}{2}) = 0 \\
\end{align*}


Все скалярные произведения равны нулю. Это означает, что система векторов \(\{v_0, v_1, v_2, v_3\}\) ортогональна в новом базисе.

Таким образом, система многочленов \(\{1, t, t^2-\frac{2}{6}, t^3-\frac{2}{5}t\}\) является ортогональной.\\


4)Для разложения многочлена \(P_3(t) = t^3+2t^2-2t+1\) по системе векторов \(\{v_0, v_1, v_2, v_3\}\), где

\(v_0 = (1, 0, 0, 0)\)

\(v_1 = (0, 1, 0, 0)\)

\(v_2 = (0, 0, \frac{3}{2}, 0)\)

\(v_3 = (0, 0, 0, \frac{5}{2})\),

мы можем представить систему векторов в виде матрицы \(V\) и вычислить коэффициенты разложения, используя матричную алгебру.

Сначала составим матрицу \(V\) из векторов системы:

\[
V = \begin{pmatrix}
1 & 0 & 0 & 0 \\
0 & 1 & 0 & 0 \\
0 & 0 & \frac{3}{2} & 0 \\
0 & 0 & 0 & \frac{5}{2} \\
\end{pmatrix}
\]

Затем вычислим обратную матрицу \(V^{-1}\):

\[
V^{-1} = \begin{pmatrix}
1 & 0 & 0 & 0 \\
0 & 1 & 0 & 0 \\
0 & 0 & \frac{2}{3} & 0 \\
0 & 0 & 0 & \frac{2}{5} \\
\end{pmatrix}
\]

Теперь вычислим коэффициенты разложения путем умножения матрицы \(V^{-1}\) на вектор многочлена \(P_3(t)\):

\[
\begin{pmatrix}
c_0 \\
c_1 \\
c_2 \\
c_3 \\
\end{pmatrix}
=
\begin{pmatrix}
1 & 0 & 0 & 0 \\
0 & 1 & 0 & 0 \\
0 & 0 & \frac{2}{3} & 0 \\
0 & 0 & 0 & \frac{2}{5} \\
\end{pmatrix}
\begin{pmatrix}
1 \\
2 \\
-2 \\
1 \\
\end{pmatrix}
\]

Выполняя умножение матриц, получаем:

\[
\begin{pmatrix}
c_0 \\
c_1 \\
c_2 \\
c_3 \\
\end{pmatrix}
=
\begin{pmatrix}
1 \\
2 \\
-\frac{4}{3} \\
\frac{4}{5} \\
\end{pmatrix}
\]

Таким образом, разложение многочлена \(P_3(t)\) по системе векторов \(\{v_0, v_1, v_2, v_3\}\) имеет вид:

\(P_3(t) = 1 \cdot v_0 + 2 \cdot v_1 - \frac{4}{3} \cdot v_2 + \frac{4}{5} \cdot v_3\) или \(P_3(t) = v_0 + 2v_1 -\frac{4}{3}v_2 + \frac{4}{5}v_3\)\\


Б)


\textbf{Задание:} Дано пространство $R$ функций, непрерывных (или имеющих конечный разрыв) на отрезке [$-\pi; \pi$], со скалярным произведением $(f,g)=\int_{-\pi}^{\pi}f(t)g(t)dt$ и длиной вектора $||f||=\sqrt{(f,f)}$. Тригонометрические многочлены $P_n(t)=\frac{a_0}{2}+a_1cos(t)+b_1sin(t)+...+a_ncos(nt)+b_nsin(nt)$, где $a_k,b_k$ - вещественные коэффициенты, образуют подпространство $P$ пространства $R$. Требуется найти многочлен $P_n(t)$ в пространстве $P$, минимально отличающийся от функции $f(t)$ - вектора пространства $R$.\\

\textbf{Решение:}\\


1) Для начала проверим ортогональность системы функций $\{1, \cos(t), \sin(t), ..., \cos(nt), \sin(nt)\}$ в пространстве $R$ с заданным скалярным произведением $(f,g)=\int_{-\pi}^{\pi} f(t)g(t)dt$.

Рассмотрим произведение двух функций $f_i(t)$ и $f_j(t)$, где $f_i(t)$ соответствует $i$-й функции из системы, а $f_j(t)$ соответствует $j$-й функции из системы:
\[(f_i, f_j) = \int_{-\pi}^{\pi} f_i(t)f_j(t)dt.\]

Проверим ортогональность для всех пар функций. Если $(f_i, f_j) = 0$, то функции $f_i(t)$ и $f_j(t)$ ортогональны.

\begin{align*}
(f_0, f_1) &= \int_{-\pi}^{\pi} 1 \cdot \cos(t) dt = 0\\
(f_0, f_2) &= \int_{-\pi}^{\pi} 1 \cdot \sin(t) dt = 0\\
(f_1, f_2) &= \int_{-\pi}^{\pi} \cos(t) \cdot \sin(t) dt = 0\\
&\vdots\\
(f_{n-1}, f_n) &= \int_{-\pi}^{\pi} \cos((n-1)t) \cdot \sin(nt) dt = 0
\end{align*}

Таким образом, все пары функций из системы ортогональны. 

Теперь нормируем систему функций. Для этого найдем длину каждой функции $f_i(t)$, определенную как $||f_i|| = \sqrt{(f_i, f_i)}$:

\begin{align*}
||f_0|| &= \sqrt{(f_0, f_0)} = \sqrt{\int_{-\pi}^{\pi} 1 \cdot 1 dt} = \sqrt{2\pi}\\
||f_1|| &= \sqrt{(f_1, f_1)} = \sqrt{\int_{-\pi}^{\pi} \cos^2(t) dt} = \sqrt{\pi}\\
||f_2|| &= \sqrt{(f_2, f_2)} = \sqrt{\int_{-\pi}^{\pi} \sin^2(t) dt} = \sqrt{\pi}\\
&\vdots\\
||f_{n-1}|| &= \sqrt{(f_{n-1}, f_{n-1})} = \sqrt{\int_{-\pi}^{\pi} \cos^2((n-1)t) dt} = \sqrt{\frac{2\pi (n-1)+sin(2\pi (n-1))}{2(n-1)}} =  \sqrt{\pi}\\
||f_n|| &= \sqrt{(f_n, f_n)} = \sqrt{\int_{-\pi}^{\pi} \sin^2(nt) dt} = \sqrt{\frac{2\pi n-sin(2\pi n)}{2n}} = \sqrt{\pi}
\end{align*}

Теперь нормируем каждую функцию, разделив ее на длину:
\begin{align*}
\frac{1}{||f_0||}f_0 &= \frac{1}{\sqrt{2\pi}}\\
\frac{1}{||f_1||}f_1 &= \frac{1}{\sqrt{\pi}}\cos(t)\\
\frac{1}{||f_2||}f_2 &= \frac{1}{\sqrt{\pi}}\sin(t)\\
&\vdots\\
\frac{1}{||f_{n-1}||}f_{n-1} &= \frac{1}{\sqrt{\pi}}\cos(nt)\\
\frac{1}{||f_n||}f_n &= \frac{1}{\sqrt{\pi}}\sin((n-1)t)
\end{align*}

Таким образом, мы получили ортогональный и нормированный базис системы функций в пространстве $R$.
Система функций $\{1, \cos(t), \sin(t), ..., \cos(nt), \sin(nt)\}$ является ортогональным базисом подпространства $P$ в пространстве $R$ со скалярным произведением $(f,g)=\int_{-\pi}^{\pi} f(t)g(t)dt$ и нормой $||f||=\sqrt{(f,f)}$.\\


2) Найдём проекции вектора $f(t)=0,5\cdot t$ на векторы полученного ортонормированного
базиса.
При нормированных векторах базиса можно упростить вычисление проекций. Проекция вектора $f(t) = 0.5 \cdot t$ на каждый из нормированных векторов базиса будет равна скалярному произведению между $f(t)$ и соответствующим вектором базиса, умноженному на базисный вектор.
\begin{equation*}
    \mathcal{P}_{e}^{\bot e}(x) = (x,e)\cdot e
\end{equation*}

Проекция вектора \(f(t) = 0.5t\) на вектор \(\frac{1}{\sqrt{2\pi}}\):
\[\mathcal{P}_{\frac{1}{\sqrt{2\pi}}}^{\bot \frac{1}{\sqrt{2\pi}}}(0.5t) = \left(0.5t, \frac{1}{\sqrt{2\pi}}\right) \cdot \frac{1}{\sqrt{2\pi}}\]


Выполним вычисления:
\[\mathcal{P}_{\frac{1}{\sqrt{2\pi}}}^{\bot \frac{1}{\sqrt{2\pi}}}(0.5t) = \frac{1}{\sqrt{2\pi}} \cdot \int_{-\pi}^{\pi} 0.5t \cdot \frac{1}{\sqrt{2\pi}} dt\]
\[\mathcal{P}_{\frac{1}{\sqrt{2\pi}}}^{\bot \frac{1}{\sqrt{2\pi}}}(0.5t) = \frac{1}{2\pi} \int_{-\pi}^{\pi} t dt\]
\[\mathcal{P}_{\frac{1}{\sqrt{2\pi}}}^{\bot \frac{1}{\sqrt{2\pi}}}(0.5t) = \frac{1}{2\pi} \cdot \left[\frac{t^2}{2}\right]_{-\pi}^{\pi}\]
\[\mathcal{P}_{\frac{1}{\sqrt{2\pi}}}^{\bot \frac{1}{\sqrt{2\pi}}}(0.5t) = \frac{1}{2\pi} \cdot \left(\frac{\pi^2}{2} - \frac{(-\pi)^2}{2}\right)\]
\[\mathcal{P}_{\frac{1}{\sqrt{2\pi}}}^{\bot \frac{1}{\sqrt{2\pi}}}(0.5t) = 0\]

Проекция вектора \(f(t) = 0.5t\) на вектор \(\frac{1}{\sqrt{\pi}}\cos(nt)\):
\[\mathcal{P}_{\frac{1}{\sqrt{\pi}}\cos(nt)}^{\bot \frac{1}{\sqrt{\pi}}\cos(nt)}(0.5t) = \left(0.5t, \frac{1}{\sqrt{\pi}}\cos(nt)\right) \cdot \frac{1}{\sqrt{\pi}}\cos(nt)\]


Выполним вычисления:
\[\mathcal{P}_{\frac{1}{\sqrt{\pi}}\cos(nt)}^{\bot \frac{1}{\sqrt{\pi}}\cos(nt)}(0.5t) = \frac{1}{\sqrt{\pi}}\cos(nt) \cdot \int_{-\pi}^{\pi} 0.5t \cdot \frac{1}{\sqrt{\pi}}\cos(nt) dt = 0\]


Продолжим с вычислением проекции вектора \(f(t) = 0.5t\) на вектор \(\frac{1}{\sqrt{\pi}}\sin(nt)\):

\[\mathcal{P}_{\frac{1}{\sqrt{\pi}}\sin(nt)}^{\bot \frac{1}{\sqrt{\pi}}\sin(nt)}(0.5t) = \left(0.5t, \frac{1}{\sqrt{\pi}}\sin(nt)\right) \cdot \frac{1}{\sqrt{\pi}}\sin(nt)\]

Выполним вычисления:

\[\mathcal{P}_{\frac{1}{\sqrt{\pi}}\sin(nt)}^{\bot \frac{1}{\sqrt{\pi}}\sin(nt)}(0.5t) = \frac{1}{\sqrt{\pi}}\sin(nt) \cdot \int_{-\pi}^{\pi} 0.5t \cdot \frac{1}{\sqrt{\pi}}\sin(nt) dt = \]
\[= \frac{(sin(\pi n)-\pi n cos(\pi n ))\cdot sin(nt)}{\pi n^2}= - \frac{(-1)^n\cdot sin(nt)}{n}\]


3) Запишем минимально отстоящий многочлен $P_n(t)$ с найденными коэффициентами
(тригонометрический многочлен Фурье для данной функции).\\
\[P_n^{min}(t)= \sum_{n=1}^{\infty}\left(- \frac{(-1)^n\cdot sin(nt)}{n}\right)\]


4) Изобразим графики $f(t)$ и многочлена Фурье различных
порядков n\\

\begin{center}
    \begin{tikzpicture}
        \begin{axis}[
          axis lines=middle,
          xlabel=$x$,
          ylabel=$y$,
          xmin=-10,
          xmax=10,
          ymin=-5,
          ymax=5,
          xtick={-10,-8,-6,-4,-2,0,2,4,6,8,10},
          ytick={-5,-4,-3,-2,-1,0,1,2,3,4,5},
          xticklabels={-10,-8,-6,-4,-2,0,2,4,6,8,10},
          yticklabels={-5,-4,-3,-2,-1,0,1,2,3,4,5},
          legend style={at={(0.05,0.95)},anchor=north west},
          width=12cm,
          height=8cm
        ]
        % Координатные оси
        \addplot[black,thick,->] coordinates {(0,0) (10,0)};
        \addplot[black,thick,->] coordinates {(0,0) (0,5)};
        
        % Функция f(x) = 0.5x
        \addplot[blue,thick,domain=-10:10,samples=100] {0.5*x};
        \addlegendentry{$f(x) = 0.5x$}
        
        % Функция g(x) = 0.5x + (-1)^n * sin(nx)/n
        \addplot[red,thick,domain=-10:10,samples=100] {0.5*x + ((-1)^5*sin(5*x))/(5)};
        \addlegendentry{$g(x) = 0.5x + ((-1)^5 \cdot \sin(5x))/5$}
        \end{axis}
      \end{tikzpicture}
      \captionof{figure}{n=5}
      \begin{tikzpicture}
        \begin{axis}[
          axis lines=middle,
          xlabel=$x$,
          ylabel=$y$,
          xmin=-10,
          xmax=10,
          ymin=-5,
          ymax=5,
          xtick={-10,-8,-6,-4,-2,0,2,4,6,8,10},
          ytick={-5,-4,-3,-2,-1,0,1,2,3,4,5},
          xticklabels={-10,-8,-6,-4,-2,0,2,4,6,8,10},
          yticklabels={-5,-4,-3,-2,-1,0,1,2,3,4,5},
          legend style={at={(0.05,0.95)},anchor=north west},
          width=12cm,
          height=8cm
        ]
        % Координатные оси
        \addplot[black,thick,->] coordinates {(0,0) (10,0)};
        \addplot[black,thick,->] coordinates {(0,0) (0,5)};
        
        % Функция f(x) = 0.5x
        \addplot[blue,thick,domain=-10:10,samples=100] {0.5*x};
        \addlegendentry{$f(x) = 0.5x$}
        
        % Функция g(x) = 0.5x + (-1)^n * sin(nx)/n
        \addplot[red,thick,domain=-10:10,samples=100] {0.5*x + ((-1)^(10)*sin(10*x))/(10)};
        \addlegendentry{$g(x) = 0.5x + ((-1)^{10} \cdot \sin(10x))/10$}
        \end{axis}
      \end{tikzpicture}
      \captionof{figure}{n=10}
      \begin{tikzpicture}
        \begin{axis}[
          axis lines=middle,
          xlabel=$x$,
          ylabel=$y$,
          xmin=-10,
          xmax=10,
          ymin=-5,
          ymax=5,
          xtick={-10,-8,-6,-4,-2,0,2,4,6,8,10},
          ytick={-5,-4,-3,-2,-1,0,1,2,3,4,5},
          xticklabels={-10,-8,-6,-4,-2,0,2,4,6,8,10},
          yticklabels={-5,-4,-3,-2,-1,0,1,2,3,4,5},
          legend style={at={(0.05,0.95)},anchor=north west},
          width=12cm,
          height=8cm
        ]
        % Координатные оси
        \addplot[black,thick,->] coordinates {(0,0) (10,0)};
        \addplot[black,thick,->] coordinates {(0,0) (0,5)};
        
        % Функция f(x) = 0.5x
        \addplot[blue,thick,domain=-10:10,samples=100] {0.5*x};
        \addlegendentry{$f(x) = 0.5x$}
        
        % Функция g(x) = 0.5x + (-1)^n * sin(nx)/n
        \addplot[red,thick,domain=-10:10,samples=100] {0.5*x + ((-1)^(15)*sin(15*x))/(15)};
        \addlegendentry{$g(x) = 0.5x + ((-1)^{15} \cdot \sin(15x))/15$}
        \end{axis}
      \end{tikzpicture}
      \captionof{figure}{n=15}
\end{center}


5) При росте порядка многочлена Фурье мы приближаемся и исходной функции $f(x)=0.5\cdot x$ и при стремлении $n$ к бесконечности многочлен совпадёт с функцией.

\section{Приведение уравнения поверхности 2-го порядка к каноническому виду}


\textbf{Задание:} Дано уравнение поверхности 2-го порядка: $3x^2-2yz=0$. Привести к каноническому виду данное уравнение.\\

\textbf{Решение:}\\


1) Матрица поверхности второго порядка 3x3 обычно задается следующим образом:

\[
A = \begin{bmatrix}
a_{11} & a_{12} & a_{13} \\
a_{21} & a_{22} & a_{23} \\
a_{31} & a_{32} & a_{33}
\end{bmatrix}
\]

Каждый элемент \(a_{ij}\) матрицы соответствует коэффициенту при соответствующем члене поверхности второго порядка. Общий вид уравнения поверхности второго порядка может быть записан как:

\[
f(x, y, z) = a_{11}x^2 + a_{22}y^2 + a_{33}z^2 + 2a_{12}xy + 2a_{13}xz + 2a_{23}yz + 2a_{21}x + 2a_{31}z + 2a_{32}y + a_{00}
\]

Здесь \(a_{00}\) представляет свободный член и обычно не включается в матрицу, поскольку он не соответствует элементу матрицы. Он определяет базовый уровень поверхности без учета влияния переменных. В контексте геометрической интерпретации, свободный член может соответствовать смещению поверхности относительно начала координат.
\\ 


Для данного уравнения \(3x^2 - 2yz = 0\) мы можем составить матрицу следующим образом:

\[
A = \begin{bmatrix}
3 & 0 & 0 \\
0 & 0 & -1 \\
0 & -1 & 0 \\
\end{bmatrix}
\]
\\


Приведём к каноническому виду уравнение при помощи метода Лагранжа.
\[
3x^2-2yz=
\begin{vmatrix}
 x=\frac{1}{\sqrt{3}}p_1\\
y=\frac{1}{\sqrt{2}}(p_2-p_3)\\
z=\frac{1}{\sqrt{2}}(p_2+p_3)\\
\end{vmatrix}=
{p_1}^2-\left(p_2-p_3\right)\left(p_2+p_3\right)={p_1}^2-{p_2}^2+{p_3}^2
\]


Диагональная матрица в каноническом виде будет следующей:\\
\[
A'=
\begin{bmatrix}
1 & 0 & 0 \\
0 & -1 & 0 \\
0 & 0 & 1 \\
\end{bmatrix}
\]
\\ \\


Для решения данной задачи методом ортогональных преобразований, нам потребуется выполнить следующие шаги:
\\

1. Найдем собственные значения и собственные векторы матрицы \(A\). Для этого решим уравнение \(\det(A - \lambda I) = 0\), где \(\lambda\) - собственное значение, а \(I\) - единичная матрица:
\[
\begin{vmatrix}
3 - \lambda & 0 & 0 \\
0 & -\lambda & -1 \\
0 & -1 & -\lambda \\
\end{vmatrix}
= 0
\]

2. Решим полученное характеристическое уравнение и найдем собственные значения \(\lambda_1 = 3\), \(\lambda_2 = 0\), \(\lambda_3 = -3\).

3. Для каждого собственного значения найдем соответствующий собственный вектор, решив систему уравнений \((A - \lambda I) \mathbf{v} = \mathbf{0}\), где \(\mathbf{v}\) - собственный вектор:
   
   Для \(\lambda_1 = 3\):
   \[
   (A - 3I) \mathbf{v}_1 = \mathbf{0} \Rightarrow
   \begin{bmatrix}
   0 & 0 & 0 \\
   0 & -3 & -1 \\
   0 & -1 & -3 \\
   \end{bmatrix}
   \begin{bmatrix}
   v_{11} \\
   v_{12} \\
   v_{13} \\
   \end{bmatrix}
   =
   \begin{bmatrix}
   0 \\
   0 \\
   0 \\
   \end{bmatrix}
   \]
   Решив эту систему, получим собственный вектор \(\mathbf{v}_1 = \begin{bmatrix} 0 \\ -1 \\ 1 \end{bmatrix}\).

   Для \(\lambda_2 = 0\):
   \[
   (A - 0I) \mathbf{v}_2 = \mathbf{0} \Rightarrow
   \begin{bmatrix}
   3 & 0 & 0 \\
   0 & 0 & -1 \\
   0 & -1 & 0 \\
   \end{bmatrix}
   \begin{bmatrix}
   v_{21} \\
   v_{22} \\
   v_{23} \\
   \end{bmatrix}
   =
   \begin{bmatrix}
   0 \\
   0 \\
   0 \\
   \end{bmatrix}
   \]
   Решив эту систему, получим собственный вектор \(\mathbf{v}_2 = \begin{bmatrix} 1 \\ 0 \\ 0 \end{bmatrix}\).

   Для \(\lambda_3 = -3\):
   \[
   (A + 3I) \mathbf{v}_3 = \mathbf{0} \Rightarrow
   \begin{bmatrix}
   6 & 0 & 0 \\
   0 & 3 & -1 \\
   0 & -1 & 3 \\
   \end{bmatrix}
   \begin{bmatrix}
   v_{31} \\
   v_{32} \\
   v_{33} \\
   \end{bmatrix}
   =
   \begin{bmatrix}
   0 \\
   0 \\
   0 \\
   \end{bmatrix}
   \]
   Решив эту систему, получим собственный вектор \(\mathbf{v}_3 = \begin{bmatrix} 0 \\ 1 \\ 1 \end{bmatrix}\).

Нормализуем найденные собственные векторы, чтобы получить ортонормированный базис. Для этого разделим каждый собственный вектор на его длину:
   \(\mathbf{u}_1 = \frac{\mathbf{v}_1}{\|\mathbf{v}_1\|}\),
   \(\mathbf{u}_2 = \frac{\mathbf{v}_2}{\|\mathbf{v}_2\|}\),
   \(\mathbf{u}_3 = \frac{\mathbf{v}_3}{\|\mathbf{v}_3\|}\).
   
\[
\mathbf{u}_1 = \frac{\mathbf{v}_1}{\|\mathbf{v}_1\|} = \frac{1}{\sqrt{2}}\begin{bmatrix} 0 \\ 1 \\ 1 \end{bmatrix}
\]

\[
\mathbf{u}_2 = \frac{\mathbf{v}_2}{\|\mathbf{v}_2\|} = \frac{1}{\sqrt{2}}\begin{bmatrix} 0 \\ 1 \\ -1 \end{bmatrix}
\]

\[
\mathbf{u}_3 = \frac{\mathbf{v}_3}{\|\mathbf{v}_3\|} = \begin{bmatrix} 1 \\ 0 \\ 0 \end{bmatrix}
\]

Теперь у нас есть ортонормированные собственные векторы \(\mathbf{u}_1\), \(\mathbf{u}_2\), \(\mathbf{u}_3\), которые составляют ортогональную матрицу преобразования \(P\).

\[
P = \begin{bmatrix}
 \begin{bmatrix} 1 \\ 0 \\ 0 \end{bmatrix} & \frac{1}{\sqrt{2}}\begin{bmatrix} 0 \\ 1 \\ 1 \end{bmatrix} & \frac{1}{\sqrt{2}}\begin{bmatrix} 0 \\ 1 \\ -1 \end{bmatrix} 
\end{bmatrix} = \begin{bmatrix}
1 & 0 & 0 \\
 0&\frac{1}{\sqrt{2}} & \frac{1}{\sqrt{2}}  \\
 0&\frac{1}{\sqrt{2}} & -\frac{1}{\sqrt{2}}
\end{bmatrix}
\]


Так как определитель матрицы, составленной из ортогонального базиса равен -1, то это означает поворот системы координат вместе с зеркальным отображением. В декартовой системе координат это является левой тройков векторов.
\\ \\
\[
\begin{bmatrix}
    1 & 0 & 0 \\
     0&\frac{1}{\sqrt{2}} & \frac{1}{\sqrt{2}}  \\
     0&\frac{1}{\sqrt{2}} & -\frac{1}{\sqrt{2}}
    \end{bmatrix}
    \]
\\В данном базисе квадратичная форма будет иметь диагональный вид.\\
Выразим старые координаты черещ новые по формуле:
\[
    X = P X'
\]
\[
    \begin{bmatrix}
        x \\
         y \\
         z
        \end{bmatrix} = \begin{bmatrix}
            1 & 0 & 0 \\
             0&\frac{1}{\sqrt{2}} & \frac{1}{\sqrt{2}}  \\
             0&\frac{1}{\sqrt{2}} & -\frac{1}{\sqrt{2}}
            \end{bmatrix} \begin{bmatrix}
                p_1 \\
                 p_2 \\
                 p_3
                \end{bmatrix}
\]
Получим систему замены переменных (что-то похожее???)
\[
    \begin{cases}
        x = p_1\\
        y = \frac{1}{\sqrt{2}}(p_2+p_3)\\
        z = \frac{1}{\sqrt{2}}(p_2-p_3)
    \end{cases}
\]
Выпишем уравнения после диагонализации:\\
\[
    3p_1^2-p_2^2+p_3^2=0
\]
Дальнейшее приведение к каноническому виду очевидно из первого пункта путём очередной замены. Так как линейной части нет, то процесс приведения к каноническому виду закончен.\\
Заметим, данный результат можно было получить не нормируя базис, а просто получить матрицу $A'$ по формуле:
\[
    A' = 
    \begin{bmatrix}
        \mathbf{v}_1\ 
        \mathbf{v}_2\
        \mathbf{v}_3
        \end{bmatrix}^{-1} \cdot A \cdot \begin{bmatrix}
            \mathbf{v}_1\ 
            \mathbf{v}_2\
            \mathbf{v}_3
            \end{bmatrix}
     \]
Где $A'$ находится очевидно после проверки полноты базиса собственного подпространства $A$ и  выглядит:\\
\[\begin{bmatrix}
    -1 & 0 & 0 \\
    0 & 1 & 0 \\
    0 & 0 & 3 \\
    \end{bmatrix}\]
Порядок собственный чисел по формулировке метода не задан\\
\\
2) Изобразим оси исходной и приведённой систем координат.\\
\begin{center}
    \begin{tikzpicture}[scale=4]
        \coordinate (O) at (0,0,0);
        \coordinate (U1) at (0,0.7,0.7);
        \coordinate (U2) at (0,0.7,-0.7);
        \coordinate (U3) at (1,0,0);

        \draw[->] (0,0,0) -- (1,0,0) node[right]{$x$};
        \draw[->] (0,0,0) -- (0,1,0) node[above]{$y$};
        \draw[->] (0,0,0) -- (0,0,1) node[below left]{$z$};

        % Vector u1
        \draw[->,red,thick] (O) -- (U1);
        \node[above right,red] at (U1) {$\mathbf{u}_1$};
        % Vector u2
        \draw[->,blue,thick] (O) -- (U2);
        \node[below left,blue] at (U2) {$\mathbf{u}_2$};
        % Vector u3
        \draw[->,green!70!black,thick] (O) -- (U3);
        \node[above right,green!70!black] at (U3) {$\mathbf{u}_3$};
        \end{tikzpicture}
        \captionof{figure}{Новый и старый базис}
\end{center}


Фигура, заданная уравнением:\\
\begin{center}
    \tdplotsetmaincoords{110}{40}
    \begin{tikzpicture}[tdplot_main_coords]
        \begin{axis}
            \addplot3 [surf, domain=-2:2, domain y=0:2*pi, shader=interp] ({x}, {sqrt(3/2)*x*cos(deg(y))}, {sqrt(3/2)*x*sin(deg(y))});
            \addplot3 [surf, domain=-2:2, domain y=0:2*pi, shader=interp] ({x}, {-sqrt(3/2)*x*cos(deg(y))}, {-sqrt(3/2)*x*sin(deg(y))});
            \end{axis}
        \end{tikzpicture}
        \captionof{figure}{Конус второго порядка}
\end{center}


\end{document}

