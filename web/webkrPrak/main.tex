\documentclass{article}
\usepackage[utf8]{inputenc} %кодировка
\usepackage[T2A]{fontenc}
\usepackage[english,russian]{babel} %русификатор 
\usepackage{mathtools} %библиотека матеши
\usepackage[left=1cm,right=1cm,top=2cm,bottom=2cm,bindingoffset=0cm]{geometry} %изменение отступов на листе
\usepackage{amsmath}
\usepackage{graphicx} %библиотека для графики и картинок
\graphicspath{}
\DeclareGraphicsExtensions{.pdf,.png,.jpg}
\usepackage{subcaption}
\usepackage{pgfplots}
\usepackage{listings}
\usepackage{xcolor}

\begin{document}
\color{darkgray}
\tiny
\begin{minipage}{.2\textwidth}
PHP скрипт, который достаёт из get имя и фамилию, выводя html
\begin{lstlisting}
<?php
if (isset($_GET['name']) 
&& isset($_GET['surname'])) {
$name = $_GET['name'];
$surname = $_GET['surname'];
echo "<h1>$name $surname!</h1>";
} else {
echo "<p>No data was given</p>";
}?>
\end{lstlisting}
\end{minipage}
\hfill
\begin{minipage}{.2\textwidth}
    Написать фрагмент кода html, блокируют контент при ш э < 1024
\begin{lstlisting}
<html><head>
<link rel="stylesheet" type=
"text/css" href="styles.css">
</head>
<body><div class="content">
<p>textext</p>
</div></body></html>
.content {display: block;}
@media (max-width: 
1023px) {.content {display:
none; }}
\end{lstlisting}
\end{minipage}
\hfill
\begin{minipage}{.2\textwidth}
    Написать JSP страницу, выводит количество пользователей, за последние 60 c
\begin{lstlisting}
<html><body><\%
Map<String, Long> requestStats = 
(ConcurrentHashMap<String, Long>)
request.getAttribute
("requestStats");
long currentTime = System
.currentTimeMillis();
int count = 0;
for (Long requestTime :
requestStats.values()) {
if (currentTime - requestTime
 <= 60000) {count++;}}\%>
<p><\%= count \%></p></body>
</html>
\end{lstlisting}
\end{minipage}
\hfill
\begin{minipage}{.2\textwidth}
Шаблон и код инициализации контекста Thymeleaf, HTML-страницу,курсы валют (доллара и евро), динамику последнюю сессию
\begin{lstlisting}
<html><body><h1>kurs val</h1>
<table><tr><th>Currency</th>
<th>Current rate</th>
<th>last trading</th></tr>
<tr th:each="currency :
${currencies}">
<td th:text="${currency.name}">
</td><td th:text="${currency
.currentRate}"></td>
<td th:text="${currency.change}">
</td></tr></table></body></html>
\end{lstlisting}
\begin{lstlisting}
@Controller
@RequestMapping("/currency")
public class CurrencyController {
@GetMapping
public String showCurrencyPage
(Model model) {
List<Currency> currencies = 
new ArrayList<>();
currencies.add(new Currency
("dollar", 75.50, -0.25));
currencies.add(new Currency
("euro", 88.20, 0.10));
model.addAttribute("currencies",
 currencies);
return "currency";}}
\end{lstlisting}
\end{minipage}
\\
\begin{minipage}{.2\textwidth}
Заменить все гиперссылки на текстовое поле со значением ссылки
\begin{lstlisting}
function replaceLinks() {
var contentDiv = document.
getElementById("content");
var links = contentDiv.get
ElementsByTagName("a");
for (var i = 0; i <
 links.length; i++) {
var link = links[i];
var textField = document.
createElement("input");
textField.type = "text";
textField.value = link.href;
link.parentNode.replaceChild
(textField, link);}}
\end{lstlisting}
\end{minipage}
\hfill
\begin{minipage}{.2\textwidth}
Написать JSP страничку, отображает корзину. 
объект - класс ShoppingItem. хранится managed bean.
\begin{lstlisting}
<% @ taglib prefix="c" uri="
http://java.sun.com/jsp/jstl/
core" %><body><h1>Cart</h1>
<table><tr><th>Name</th>
<th>Price</th></tr>
<c:forEach items='${shoppin
gCartBean.items}' var='item'>
<tr><td>${item.name}</td>
<td>${item.price}</td></tr>
</c:forEach></table></body>
@ManagedBean
public class ShoppingCartBean
{private List<ShoppingItem> 
items = new ArrayList<>();
public List<ShoppingItem> 
getItems(){return items;}
public void addItem
(ShoppingItem item){
items.add(item);}}
\end{lstlisting}
\end{minipage}
\hfill
\begin{minipage}{.2\textwidth}
Написать структуру HTTP запроса, отправляющего логин и пароль пользователя
\begin{lstlisting}
POST /login HTTP/1.1
Host: example.com
Content-Type: 
application/x-www-form-
urlencoded
Content-Length: 24
username=123&password=123
\end{lstlisting}
\end{minipage}
\hfill
\begin{minipage}{.2\textwidth}
Написать AJAX запрос который получает JSON и выводит его элементы
\begin{lstlisting}
fetch('https://example.com')
.then(response => response
.data)\n .then(data => {
for(i of data) {
let el = document.creat
eElement("span");
el.innerHTML = JSON.str
ingify(i);
document.getElementById("js
on-data").append(el)}})
.catch(error => {
console.error(error);
});
\end{lstlisting}
\end{minipage}
\\
\begin{minipage}{.2\textwidth}
FreeMarker страничка с оценками студентов, sort по времени получения оценки
\begin{lstlisting}
public static String 
generateStudentGradePage(List
<StudentGrade> studentGrades){
Configuration cfg = new 
Configuration(Configuration.
VERSION***);
cfg.setClassForTemplateLoading
(StudentGradePageGenerator.
class, "/");Template template =
cfg.getTemplate("student_grades
_template.ftl");
Map<String, Object> dataModel =
new HashMap<>();
dataModel.put("studentGrades",
studentGrades);
StringWriter writer = new 
StringWriter();
template.process(dataModel, writer);
return writer.toString();}
<!DOCTYPE html>
<html><body><table><tr>
<th>stud</th>
<th>grade</th>
<th>time</th></tr>
<#list studentGrades as grade>
<tr><td>${grade.studentName}</td>
<td>${grade.grade}</td>
<td>${grade.timestamp?str
ing("dd.MM.yyyy HH:mm:ss")}</td>
</tr></#list></table>
</body></html>
\end{lstlisting}
\end{minipage}
\hfill
\begin{minipage}{.2\textwidth}
Написать правило на CSS, что у всех посещённых ссылок жёлтый фон, кроме тех, кто лежит в news
\begin{lstlisting}
a:visited {background-color:
 yellow;}
.news a:visited {
background-color: initial;}
\end{lstlisting}
\end{minipage}
\hfill
\begin{minipage}{.2\textwidth}
CSS правило: рисовать границу в 1 пиксель для картинок в (class = ”news”) при наведении курсора
\begin{lstlisting}
.news img:hover {
border: 1px solid #000;}
\end{lstlisting}
\end{minipage}
\hfill
\begin{minipage}{.2\textwidth}
Написать сценарий, который будет считать
 количество слов «де-факто» во всех тегах div class= lecture. 
 Ещё обязательно jQuery.
\begin{lstlisting}
$(document).ready(function(){
var totalWordCount = 0;
$("div.lecture").each(function
(){var text = $(this).text();
var wordCount = text.split
("de-fakto").length - 1; 
totalWordCount +=wordCount;});
console.log(totalWordCount);
});
\end{lstlisting}
\end{minipage}
\\
\begin{minipage}{.2\textwidth}
Написать js функцию, которая заменяет содержимое <div> 
с именем класса “nyan” на изображение по ссылке
\begin{lstlisting}
function replaceNya
nContent(url) {
document.querySelec
tor('.nyan').innerHTML 
= `<img src="${URL}">`;
}
replaceNyanContent();
\end{lstlisting}
\end{minipage}
\hfill
\begin{minipage}{.2\textwidth}
Написать сервлет, который принимает из http запроса параметр name и выводит его.
\begin{lstlisting}
@WebServlet("/hello")
public class HelloServlet 
extends HttpServlet {
protected void doGet(
HttpServletRequest request,
HttpServletResponse response){
String name = request.
getParameter("name");
response.setContentType
("text/html");response.
setCharacterEncoding("UTF-8");
String htmlResponse = 
"<html><body>";if (name !=
 null && !name.isEmpty()) {
htmlResponse += "<h2>Hello, 
" + name + "!</h2>";
} else {htmlResponse += 
"<h2>Hola!</h2>";}
htmlResponse += "</body>
</html>";response.getWriter()
.write(htmlResponse);}}
\end{lstlisting}
\end{minipage}
\hfill
\begin{minipage}{.2\textwidth}
LESS / SASS / SCSS.Если параметр не обнаружен то вывести Anonymous user
\begin{lstlisting}
<!DOCTYPE html><html>
<head><title>Check 
Parameter</title>
</head><body>
<script>
var parameterValue = 
new URL(window.locat
ion.href).searchPara
ms.get("paramName");
if (!parameterValue) {
document.write("Anonymous 
user in HTML");
} </script>
</body></html>
\end{lstlisting}
\end{minipage}
\hfill
\begin{minipage}{.2\textwidth}
Написать конфигурацию сервлета (org.xxx.MyServlet) с помощью аннотации. Сервлет должен принимать все запросы от файлов .html .xhtml
\begin{lstlisting}
@WebServlet(name = "MyServlet", 
urlPatterns = {"*.html", 
"*.xhtml"})public class MyServlet
extends HttpServlet {}
\end{lstlisting}
\end{minipage}
\\
\begin{minipage}{.2\textwidth}
Напишите js функ. которая заменит все текстовые поля на кнопки с тем же текстом
\begin{lstlisting}
function replaceText(){
document.querySelectorAll(
'input[type="text"]').forEach(
function(textField) {var button
=document.createElement('button');
button.innerHTML=textField.value;
textField.parentNode.replaceChild
(button, textField);});}
replaceTextFieldsWithButtons();
\end{lstlisting}
\end{minipage}
\hfill
\begin{minipage}{.2\textwidth}
Написать PHP скрипт который формирует форму с логиным и паролем, которая при сабмите отправляет - authenticate.php посредством SuperAgent, и успешно перенаправить пользователя на main.php
\begin{lstlisting}
<form id="login-form">
<label for="username">Log:
</label><input type="text" 
id="username" name="username" 
required><label for="password">
Pass:</label><input 
type="password" id="password" 
name="password"><button type=
"submit">Log</button>
</form><script>
document.getElementById(
'login-form').addEvent
Listener('submit', function(e) 
{e.preventDefault();const
username = document.
getElementById('username').
value;const password = 
document.getElementById(
'password').value;
superagent.post('authenticate
.php').send({ username, 
password }).end(function 
(err, res) {if (res.status
=== 200) {window.location.
href = 'main.php';} else {
alert('Error');}});});
</script></body></html>
<?php if ($_SERVER['REQUEST
_METHOD'] === 'POST') {
$username = $_POST['username'];
$password = $_POST['password'];
if ($username === 'login' && 
$password === 'password') {
http_response_code(200);
exit;} else {http_response
_code(401);exit;}}?>    
\end{lstlisting}
\end{minipage}
\hfill
\begin{minipage}{.2\textwidth}
Повернуть все картинки на 90 градусов в форме с id=”dhdhgd
\begin{lstlisting}
form#dhdhgd  img {transform:
rotate(90deg);}
\end{lstlisting}
\end{minipage}
\hfill
\begin{minipage}{.2\textwidth}
Реализовать функцию на JavaScript,которая будет закрывать текущее окно,если в нем открыт https://www.google.ru
\begin{lstlisting}
if (window.location.href 
==="https://www.google.ru"){
window.close();} 
\end{lstlisting}
\end{minipage}
\\
\begin{minipage}{.2\textwidth}
Функция JavaScript, запрещающая для всех текстовых полей ввод символов, если они не латинские буквы или не цифры
\begin{lstlisting}
document.addEventListener
('input', function(e){
const inputElement = e.target;
if (inputElement.tagName === 
'INPUT' && inputElement.type 
=== 'text') {const inputValue 
= inputElement.value;
const regex = /^[a-zA-Z0-9]+$/;
if (!regex.test(inputValue)) {
inputElement.value = inputValue
.replace(/[^a-zA-Z0-9]+/g, '');}}
});  
\end{lstlisting}
\end{minipage}
\hfill
\begin{minipage}{.2\textwidth}
html форма которая отправляет ответ на вопрос из теста, при этом должен отправляться номер ответа (a,b,c..f) и номер вопроса
\begin{lstlisting}
<html><body><form action="validator
.php" method="post"><h2>Q 1</h2>
<label for="q1a"><input type=
"radio" name="question1" value=
"a" id="q1a"> a)</label><br>
<label for="q1b"><input type=
"radio" name="question1" 
value="b" id="q1b"> b)</label>
<label for="q1c"><input type=
"radio" name="question1" value
="c" id="q1c"> c)</label><br>
<h2>Q 2</h2><label for="q2a">
<input type="radio" name=
"question2" value="a" id=
"q2a"> a)</label><br>
<label for="q2b"><input type=
"radio" name="question2" 
value="b" id="q2b"> b)</label>
<label for="q2c"><input type
="radio" name="question2" value
="c" id="q2c"> c)</label><br>
<input type="submit" value="Submit">
</form></body></html>    
\end{lstlisting}
\end{minipage}
\hfill
\begin{minipage}{.2\textwidth}
Написать код сервлета, который будет перенаправлять все запросы на https://google.com
\begin{lstlisting}
@WebServlet("/redirect")
public class RedirectServlet 
extends HttpServlet {
protected void doGet(HttpServ
letRequest request, HttpServlet
Response response) throws Servle
tException, IOException {
response.setStatus(HttpServletR
esponse.SC_FOUND);
response.setHeader("Location", 
"https://www.google.com");}}
\end{lstlisting}
\end{minipage}
\hfill
\begin{minipage}{.2\textwidth}
Написать HTML страницу и сервлет, возвращающий странице количество сессий
\begin{lstlisting}
<html><body>
<p id="sessionCount">active 
sessions</p><script>
fetch('SessionCounterServlet')
.then(response => response.
text())\n.then(data => {
document.getElementById('sessi
onCount').textContent = `Active
 session count: ${data}`;})
.catch(error => console.error
(error));</script></body></html>
public class SessionCounterServlet
extends HttpServlet {
protected void doGet(HttpServletRe
quest req, HttpServletResponse 
res)throws IOException {
int activeSessions = 0;
HttpSessionContext sessionContext 
= req.getSession().getServlet
Context().getSessionContext();
if (sessionContext != null) {
activeSessions = sessionContext.
getActiveSessions().size();}
res.setContentType("text/plain");
res.getWriter().write(Integer.
toString(activeSessions));}}
\end{lstlisting}
\end{minipage}
\\
\begin{minipage}{.2\textwidth}
Страница JSP, проверяющая есть ли jsessionid в запросе и если нету - выводящая сообщение об ошибке с кодом ошибки.
\begin{lstlisting}
<html><head><meta charset=
"UTF-8"><title>jsessionid 
check</title></head><body>
<%String jsessionid = requ
est.getParameter("jsessionid");
if (jsessionid == null || 
jsessionid.isEmpty()) {
response.setStatus(HttpSer
vletResponse.SC_BAD_REQUEST);
out.println("<h1>Error: no jse
ssionid</h1>");} else {
out.println("<h1>jsessionid i
n request: " + jsessionid + 
"</h1>");}%></body></html>  
\end{lstlisting}
\end{minipage}
\hfill
\begin{minipage}{.2\textwidth}
Написать css чтоб по клику все ссылки кроме внутри h1 подчеркивались
\begin{lstlisting}
a {text-decoration: none; }
h1 a {text-decoration: underline;} 
\end{lstlisting}
\end{minipage}
\hfill
\begin{minipage}{.2\textwidth}
Написать страницу на HTML и CSS, скрывающую содержимое от пользователя и показать сообщение ”разрешение не поддерживается если разрешение экрана < 1400.
\begin{lstlisting}
<html><head><meta charset="
UTF-8"><title>Checking Scre
en Resolution</title>
<style>.content {display: 
block;}.message{display: 
none;}
@media screen and (max
-width: 1400px) {
.content {display: 
none;}
.message {display: 
block;}}</style></head>
<body><div class="cont
ent"><h1>Welcome</h1>
<p>content</p></div>
<div class="message">
<p>Your screen resolu
tion is not supported.</p>
</div></body></html> 
\end{lstlisting}
\end{minipage}
\hfill
\begin{minipage}{.2\textwidth}
Код фильтра запросов,запрещающий доступ к приложению неавторизованным пользователям (у неавторизованного пользователя в запросе отсутствует заголовок ”X-Application-User”)
\begin{lstlisting}
public class Authorizat
ionFilter implements Filter {
public void doFilter(ServletRe
quest request, ServletResponse 
response, FilterChain chain) 
throws IOException, ServletEx
ception { HttpServletRequest 
httpRequest = (HttpServletReq
uest) request; HttpServletRes
ponse httpResponse = (HttpSer
vletResponse) response;
String applicationUserHeader 
= httpRequest.getHeader("X-Appl
ication-User");if (application
UserHeader == null || applicat
ionUserHeader.isEmpty()) {
httpResponse.sendError(HttpServ
letResponse.SC_UNAUTHORIZED, 
"block");} else {chain.
doFilter(request, response);}}}
\end{lstlisting}
\end{minipage}
\\
\begin{minipage}{.2\textwidth}
Написать калькулятор на PHP
\begin{lstlisting}
<?php if ($_SERVER['REQUEST_METHOD
'] === 'POST') {
$num1 = $_POST['num1'];
$num2 = $_POST['num2'];
$operation = $_POST['operation'];
if (is_numeric($num1) && is_numeric
($num2)) {$num1 = floatval($num1);
$num2 = floatval($num2);
switch ($operation) {case 
'add':$result = $num1 + $num2; break;
case 'subtract': $result = $num1 
- $num2;break; case 'multiply':
$result = $num1 * $num2; break;
case 'divide':if ($num2 != 0) {
$result = $num1 / $num2;
} else {$result = 'Division by zero';
}break; default: $result = 'Invalid 
operation';break;}echo $result;
} else {echo 'Please enter numbers 
to calculate.';}}?>
\end{lstlisting}
\end{minipage}
\end{document}

% \newpage
% 21312
% \begin{minipage}{.2\textwidth}
%     Напишите js функ. которая заменит все текстовые поля на кнопки с тем же текстом
% \begin{lstlisting}
%     function replaceTextFieldsWithButtons() {
%         document.querySelectorAll('input[type="text"]').forEach(function(textField) {
%           var button = document.createElement('button');
%           button.textContent = textField.value;
%           textField.parentNode.replaceChild(button, textField);
%         });
%       }
%       replaceTextFieldsWithButtons();
% \end{lstlisting}
% \end{minipage}
% \hfill
% \begin{minipage}{.2\textwidth}

% \begin{lstlisting}
    
      
% \end{lstlisting}
% \end{minipage}
% \hfill
% \begin{minipage}{.2\textwidth}

% \begin{lstlisting}

% \end{lstlisting}
% \end{minipage}
% \hfill
% \begin{minipage}{.2\textwidth}

% \begin{lstlisting}

% \end{lstlisting}
% \end{minipage}
\end{document}