
\section[Термины]{Термины}
\subsection[Модель требований FRUPS+]{Модель требований FRUPS+}
Подробное описание того что должно быть реализовано системой, но при этом не должно описывать, как его нужно реализовать.
Включает в себя функциональные требования, нефункциональные. 

Функциональные определяют, что система должна делать: наборы функциональных требований (FR + номер), возможности ПО (СAP + номер), требования к безопасности (SEC + номер).
Наборы - набор свойств продукты необходимый для выполнения конкретной деятельности (сис должна обеспечивать ввод, модификацию и удаление данных о клиенте).
Требования к безопасности - метод аунтификаци, список ролей, шифрование, хранение данных в защищённых источниках (сис должна обеспечивать двухфакторную аутентификацию пользователей с помощью имени пользователя и пароля и подтверждения с помощью СМС.)

Нефункциональные: 

Usaility - учёт особенностей пользователя (быстрота ответа в интервале), эстетические требавания (цвет, расстояния между элементами), согласованность пользовательсткого интерфейса, согласованность пользовательского интерфейса, требования к справочной подсистеме, требования к пользовательской документации, требования к учебным материалам.

Reliability - частота и обработка заказов, способность системы восстанавливать продуктивное функционирование, предсказуемость поведения, точность, среднее время между отказами. Требования к надёжности, которые предназначены для способности ПО безотказно функционировать. 
В требованиях обычно указывается допустимое число отказов и сбоев за определённый промежуток времени. Способность системы восстанавливать продуктивное функционирование в течение заданного времени.
Требованием является accuracy - точность, например, проведения вычислений. Коэффициент готовности системы — отношение времени исправной работы к сумме времён исправной работы и вынужденных простоев объекта, взятых за один и тот же календарный срок.

Performance - скорость решения задач, эффективность, готовность системы к решению задач, пропускная способность, время отклика, время восстановления, использование системных ресурсов.
Требованием является скорость решения вычислительных задач. Также скорость важна в длительных инженерных расчетах, когда необходимо выполнить, например, моделирование за разумное для человека время.
Требования к эффективности фиксируют процент времени, которое тратится на выполнение полезных задач, по отношению к времени на выполнение общесистемных.
Требованием к производительности является готовность (availability) быстро начать выполнение задачи.
Какой объём данных или запросов система может обработать за единицу времени.
Для большой реактивности придется пожертвовать пропускной способностью.

Supportability - способность системы к расширению и масштабированию и выполнению большего объема обработки данных. Адаптируемость под конкретные задачи, поддерживаемость.
Требования к совместимости позволяют использовать различные операционные системы, версии продуктов, браузеров и пр. совместно с разработанным ПО. Отдельно выделяются системные требования и минимальные требования к установке системы, например, объём
ОЗУ, количество и частота процессоров и пр.

\subsection[Доменная модель]{Доменная модель}
Доменная модель — это концептуальная модель предметной области, которая отображает ключевые сущности, их атрибуты и взаимосвязи между ними, а также основные правила бизнес-логики. Эта модель помогает разработчикам и всем участникам проекта лучше понять структуру и функционирование системы, на которой они работают.

