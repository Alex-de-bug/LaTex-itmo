\documentclass{article}
\usepackage[utf8]{inputenc} %кодировка
\usepackage[T2A]{fontenc}
\usepackage[english,russian]{babel} %русификатор 
\usepackage{mathtools} %библиотека матеши
\usepackage[left=1cm,right=1cm,top=2cm,bottom=2cm,bindingoffset=0cm]{geometry} %изменение отступов на листе
\usepackage{amsmath}
\usepackage{graphicx} %библиотека для графики и картинок
\graphicspath{}
\DeclareGraphicsExtensions{.pdf,.png,.jpg}
\usepackage{subcaption}
\usepackage{pgfplots}
\usepackage{tikz}

\begin{document}

\begin{tikzpicture}
    % Оси координат
    \draw[->] (4.6,0) -- (28,0) node[right] {$x$};
    \draw[->] (0,0) -- (0,10) node[above] {$y$};
    
    % Разметка оси x с интервалом 0.1
    \foreach \x in {5,10,...,27} {
      \draw (\x/5,-0.2) -- (\x/5,0.2) node[below] {\x};
    }
    
    % Разметка оси y с интервалом 0.2
    \foreach \y in {2,4,...,10} {
      \draw (-0.2,\y/5) -- (0.2,\y/5) node[left] {\y};
    }
    
    % Точки (умножены на 5)
    \foreach \x/\y in {4.64/0.433, 4.765/0.935, 4.885/1.472, 5.015/1.756, 5.145/1.51, 5.27/0.958, 5.4/0.433} {
      \filldraw (\x*5,\y*5) circle (2pt);
    }
    
    % Соединяем точки плавными кривыми (умножены на 5)
    \draw[thick, smooth] plot coordinates {(4.64*5,0.433*5) (4.765*5,0.935*5) (4.885*5,1.472*5) (5.015*5,1.756*5) (5.145*5,1.51*5) (5.27*5,0.958*5) (5.4*5,0.433*5)};
  \end{tikzpicture}

\end{document}
