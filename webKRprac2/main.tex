\documentclass{article}
\usepackage[utf8]{inputenc} %кодировка
\usepackage[T2A]{fontenc}
\usepackage[english,russian]{babel} %русификатор 
\usepackage{mathtools} %библиотека матеши
\usepackage[left=1cm,right=1cm,top=1cm,bottom=0cm,bindingoffset=0cm]{geometry} %изменение отступов на листе
\usepackage{amsmath}
\usepackage{graphicx} %библиотека для графики и картинок
\graphicspath{}
\DeclareGraphicsExtensions{.pdf,.png,.jpg}
\usepackage{subcaption}
\usepackage{pgfplots}
\usepackage{xcolor}
\usepackage{amsmath}
\usepackage{graphicx} %библиотека для графики и картинок
\graphicspath{}
\DeclareGraphicsExtensions{.pdf,.png,.jpg}
\usepackage{subcaption}
\usepackage{pgfplots}
\usepackage{amssymb}
\usepackage{physics}
\usepackage{listings}
\usepackage{fancyvrb}
% \newcommand{\bil}[2]{%
% \begin{minipage}[t]{0.3\textwidth}
%     \textbf{#1} \\
%     \begin{lstlisting}[language=Java, frame=single, basicstyle=\tiny\ttfamily, breaklines=true, breakatwhitespace=true, postbreak=\mbox{\textcolor{red}{$\hookrightarrow$}\space}]
%         #2
%     \end{lstlisting}
% \end{minipage}%
% }
\lstset{basicstyle=\tiny\ttfamily}

% Define JavaScript language



\begin{document}
\begin{center}
    \LaTeX
\end{center}
\tiny


\begin{minipage}[t]{0.3\textwidth}
    \textbf{Написать исходный код CDI бина, реализующего паттерн «команда» (Command Pattern)}
    \begin{lstlisting}[frame=single, basicstyle=\tiny\ttfamily, breaklines=true, breakatwhitespace=true, postbreak=\mbox{\textcolor{red}{$\hookrightarrow$}\space}]
interface Command {void execute();}
@Named(value="cmd1") @ApplicationScoped
public class Cmd1 implements Command {
    void execute() { ... };}
@Named(value="cmd2") @ApplicationScoped
public class Cmd2 implements Command {
    void execute() { ... };}
@Named @ApplicationScoped
public class MyBean implements Command {
    private final Command cmd1, cmd2;
    @Inject
    public MyBean(@Named("cmd1") Command cmd1, @Named("cmd2") Command cmd2) { this.cmd1 = cmd1; this.cmd2 = cmd2;}
    public void cmd(int n) {if (n == 1) cmd1.execute(); if (n == 2) cmd2.execute();}}
    \end{lstlisting}
\end{minipage}%
\hfill
\begin{minipage}[t]{0.3\textwidth}
    \textbf{JSF страница, динамически подгружаемая и выводящая новостную ленту с новостями формата: автор, заголовок, дата, иллюстрация, аннотация и полный текст(показывается при нажатии на соответствующую строчку)}
    \begin{lstlisting}[frame=single, basicstyle=\tiny\ttfamily, breaklines=true, breakatwhitespace=true, postbreak=\mbox{\textcolor{red}{$\hookrightarrow$}\space}]
<h:body> <h:dataTable value="#{newsBean.newsList}" var="news" border="1">
<h:column> <f:facet name="header">Author</f:facet>
<h:outputText value="#{news.author}" /> </h:column>
<h:column> <f:facet name="header">Title</f:facet>
<h:outputText value="#{news.title}" /> </h:column>
<h:column> <f:facet name="header">Date</f:facet>
<h:outputText value="#{news.date}" /> </h:column>
<h:column> <f:facet name="header">Image</f:facet>
<h:graphicImage value="#{news.image}" width="100" height="100" alt="News Image" />
</h:column> <h:column> <f:facet name="header">Summary</f:facet>
<h:outputText value="#{news.summary}" /> </h:column>
<h:column> <f:facet name="header">Full Text</f:facet>
<h:commandLink value="Show Full Text" action="#{newsBean.showFullText(news)}">
<f:setPropertyActionListener target="#{newsBean.selectedNews}" value="#{news}" />
</h:commandLink> </h:column> </h:dataTable>
<h:panelGroup rendered="#{not empty newsBean.selectedNews}">
<h3>Full Text of News:</h3>
<h:outputText value="#{newsBean.selectedNews.fullText}" escape="false" />
</h:panelGroup> </h:body>
    \end{lstlisting}
\end{minipage}%
\hfill
\begin{minipage}[t]{0.3\textwidth}
    \textbf{Написать React компонент формирующий таблицу пользователей, данные приходят в props}
\begin{lstlisting}[frame=single, basicstyle=\tiny\ttfamily, breaklines=true, breakatwhitespace=true, postbreak=\mbox{\textcolor{red}{$\hookrightarrow$}\space}]
function UsersTable(props) { return (
<table className="table"> <thead>
<tr><td>Name</td><td>Surname</td></tr>
</thead> <tbody> { props.data.map((user, i) => (
<tr key={i}> <td>{user.name}</td>
<td>{user.surname}</td> </tr>))} </tbody> </table>)}
\end{lstlisting}
\end{minipage}%
\\

\begin{minipage}[t]{0.3\textwidth}
    \textbf{Написать веб-приложение на JSF (xhtml + CDI-бин) со списком студентов и бин, который будет реализовывать логику отчисления студентов. Напротив каждого имени студента должна быть кнопка "отчислить". Обновление должно производиться при помощи AJAX}
    \begin{lstlisting}[frame=single, basicstyle=\tiny\ttfamily, breaklines=true, breakatwhitespace=true, postbreak=\mbox{\textcolor{red}{$\hookrightarrow$}\space}]
@Named
@ApplicationScoped
public class StudentBean implements Serializable {
private List<String> students;
public List<String> getStudents(){ return students;}
public void expelStudent(String studentName) {
    students.remove(studentName);}}
    
<h:body><h:form>
<ui:repeat value="#{studentBean.students}" var="student"> #{student}
<h:commandButton value="Expel" action="#{stude ntBean.expelStudent(student)}">
<f:ajax execute="@this" render="@form" />
</h:commandButton>
<br/></ui:repeat></h:form></h:body>
    \end{lstlisting}
\end{minipage}%
\hfill
\begin{minipage}[t]{0.3\textwidth}
    \textbf{Интерфейс на React, формирующий две страницы с разными URL: Главную (/home) и Новости (/news). Переход между страницами должен осуществляться посредством гиперссылок.}
    \begin{lstlisting}[ frame=single, basicstyle=\tiny\ttfamily, breaklines=true, breakatwhitespace=true, postbreak=\mbox{\textcolor{red}{$\hookrightarrow$}\space}]
export function App(props) {
return (<BrowserRouter>
<Routes><Route path="/home" element={<div><h1>Home</h1> <a href='/news' > news</a></div>} />
<Route path="/news" element={<div><h1>news</h1> <a href='/home' > home</a></div>} />
</Routes></BrowserRouter>);}
    \end{lstlisting}
\end{minipage}%
\hfill
\begin{minipage}[t]{0.3\textwidth}
    \textbf{Написать CDI Bean калькулятор, поддерживающий 4 базовые операции для целых чисел}
    \begin{lstlisting}[frame=single, basicstyle=\tiny\ttfamily, breaklines=true, breakatwhitespace=true, postbreak=\mbox{\textcolor{red}{$\hookrightarrow$}\space}]
@Named(value="calc")
@ApplicationScoped
public class Calc implements Serializeable{
public int add(int a, int b) { return a + b; }
public int sub(int a, int b) { return a - b; }
public int mul(int a, int b) { return a * b; }
public int div(int a, int b) { return a / b; }}
    \end{lstlisting}
\end{minipage}%
\\

\begin{minipage}[t]{0.3\textwidth}
    \textbf{Интерфейс JSF (xhtml страница + CDI), реализующий ввод паспортных данных (серия, номер, дата, место выдачи)}
    \begin{lstlisting}[frame=single, basicstyle=\tiny\ttfamily, breaklines=true, breakatwhitespace=true, postbreak=\mbox{\textcolor{red}{$\hookrightarrow$}\space}]
@ManagedBean @SessionScoped @Setter @Getter 
public class PassportInputBean { 	
private String series; private String number; 	
private Date date; private String place; 	
@ManagedProperty(value = "#{repoBean}") 
private PassportRepo repo; 	
public void storePassportData() { 		
repo.store(series, number, date, place);}}
<h:form>Enter series	
<h:inputText type="text" value="#{passportInputBean.series}"
required="true"/>Enter number	
<h:inputText type="text" value="#{passportInputBean.number}"> 
Enter date<h:inputText value="#{passportInputBean.date}">
<f:convertDateTime pattern="yyyy-MM-dd"/> 	
</h:inputText> Enter born place 	
<h:inputText type="text" value="#{passportInputBean.place}">
<h:commandButton value="Send" action="#{passportInputBean.storePas
sportData()}"> </h:form>
    \end{lstlisting}
\end{minipage}%
\hfill
\begin{minipage}[t]{0.3\textwidth}
    \textbf{Привести фрагмент кода управляемого бина, увеличивающего на 1 значение, отображаемое на кнопке при каждом клике по ней}
    \begin{lstlisting}[frame=single, basicstyle=\tiny\ttfamily, breaklines=true, breakatwhitespace=true, postbreak=\mbox{\textcolor{red}{$\hookrightarrow$}\space}]
@ManagedBean @ApplicationScoped
public class MyBean implements Serializable {
private int value = 0; public void increment() {
value++;} public int getValue() {return value;}
public void setValue(int value) {this.value = value;}}
<h:commandButton action= "#{myBean.increment}" value = "#{bean.value}"/>
    \end{lstlisting}
\end{minipage}%
\hfill
\begin{minipage}[t]{0.3\textwidth}
    \textbf{Сделать бин который показывает время в минутах со старта сервера}
    \begin{lstlisting}[frame=single, basicstyle=\tiny\ttfamily, breaklines=true, breakatwhitespace=true, postbreak=\mbox{\textcolor{red}{$\hookrightarrow$}\space}]
@Named("serverTimer") @ApplicaitonScoped
public class ServerTimer { private long start;
public ServerTimer() {start = System.currentTimeMillis();}
public long getTime() {return (System.currentTimeMillis() - start) / 60000;}}
    \end{lstlisting}
\end{minipage}%
\\ 

\begin{minipage}[t]{0.3\textwidth}
    \textbf{Написать страницу JSF, которая бы выводила сначала 10 простых чисел, а затем с помощью Ajax запроса ещё 10.}
    \begin{lstlisting}[frame=single, basicstyle=\tiny\ttfamily, breaklines=true, breakatwhitespace=true, postbreak=\mbox{\textcolor{red}{$\hookrightarrow$}\space}]
@ApplicationScoped @Named public class PrimeNumber{@Getter
private final List<Long> primes = new ArrayList<>();
public PrimeNumber(){nextPrimes();}
public nextPrimes(){//add in primes next 10 prime numbers}}

<h:dataTable id="table" value"#{primeNumber.primers}" var="prime">
<h:column> <h:outputText value="#{prime}"/>
</h:column> </h:dataTable> <h:form>
<h:commandButton value="Next 10" action="#{primeNumber.nextPrimes}">
<f:ajax execute="@form" render="table"/>
</h:commandButton> </h:form>
    \end{lstlisting}
\end{minipage}%
\hfill
\begin{minipage}[t]{0.3\textwidth}
    \textbf{JSF Manager Bean, после инициализации HTTP-сессии формирующий коллекцию с содержимым таблицы Н\_УЧЕБНЫЕ\_ПЛАНЫ. Для доступа к БД необходимо использовать JDBC-ресурс jdbc/OrbisPool.}
    \begin{lstlisting}[frame=single, basicstyle=\tiny\ttfamily, breaklines=true, breakatwhitespace=true, postbreak=\mbox{\textcolor{red}{$\hookrightarrow$}\space}]
@Named("myBean") @SessionScoped class MyBean {
private List<String> plans;
private List<String> getPlans() {return plans;}
@Resource(name="jdbc/OrbisPool",type=DataSource.class)
private DataSource dataSource; @PostConstruct
private void loadPlans() {
try (var conn = dataSource.getConnection()) {
var stmt = conn.createStatement();
var rs = stmt.executeQuery("SELECT name FROM plans");
plans = new ArrayList<>(); while (rs.next()) {
plans.add(rs.getString("name"));}}}}
    \end{lstlisting}
\end{minipage}%
\hfill
\begin{minipage}[t]{0.3\textwidth}
    \textbf{Написать managed bean и задать ему scope такой же как у бина otherBean}
    \begin{lstlisting}[frame=single, basicstyle=\tiny\ttfamily, breaklines=true, breakatwhitespace=true, postbreak=\mbox{\textcolor{red}{$\hookrightarrow$}\space}]
@ManagedBean @ApplicationScoped
public class OtherBean {
    @ManagedProperty(value="#{myBean}")
@Getter private MyBean myBean;}
@ManagedBean @NoneScoped @Named
public class MyBean{}
    \end{lstlisting}
\end{minipage}%
\\

\begin{minipage}[t]{0.3\textwidth}
    \textbf{Vue.js простейший чат бот, который на любое сообщение отвечает сам дурак}
    \begin{lstlisting}[frame=single, basicstyle=\tiny\ttfamily, breaklines=true, breakatwhitespace=true, postbreak=\mbox{\textcolor{red}{$\hookrightarrow$}\space}]
<template> <input v-model="message_value" placeholder="message"/> 
<button @click="send">send</button> 
<li v-for="item in message_history" :key="item"> 
{{item}}</li></template> 
<script> export default { data() { return { 
message_value: '', message_history: []}},methods: { 
send(){this.message_history.push(this.message_value); 
this.message_history.push("sam durak"); 
this.message_value = "";}}}</script>
    \end{lstlisting}
\end{minipage}%
\hfill
\begin{minipage}[t]{0.3\textwidth}
    \textbf{RestController, который реализует перевод градусов Цельсия в Фаренгейты и обратно}
    \begin{lstlisting}[frame=single, basicstyle=\tiny\ttfamily, breaklines=true, breakatwhitespace=true, postbreak=\mbox{\textcolor{red}{$\hookrightarrow$}\space}]
@RestController class TempController {
@GetMapping("/convert/c/f")
double cToF(@RequestParam double c) {return c * 1.8 + 32.0;}
@GetMapping("/convert/f/c")
double fToC(@RequestParam double f) {return (f - 32.0) / 1.8;}}
    \end{lstlisting}
\end{minipage}%
\hfill
\begin{minipage}[t]{0.3\textwidth}
    \textbf{Написать JSF страницу с многострочным полем, в которое можно вводить только строчные символы латиницы.}
    \begin{lstlisting}[frame=single, basicstyle=\tiny\ttfamily, breaklines=true, breakatwhitespace=true, postbreak=\mbox{\textcolor{red}{$\hookrightarrow$}\space}]
<h:head> <script> function validateInput() {
var textArea = document.getElementById('inputTextArea');
var regex = /^[a-z\s]*$/; if (!regex.test(textArea.value)) {
textArea.value = '';}}</script></h:head>
<h:body><h:form>
<h:inputTextarea id="inputTextArea" rows="5" cols="30" oninput="validateInput()" /></h:form></h:body>
    \end{lstlisting}
\end{minipage}%

\end{document}





\begin{minipage}[t]{0.3\textwidth}
    \textbf{}
    \begin{lstlisting}[frame=single, basicstyle=\tiny\ttfamily, breaklines=true, breakatwhitespace=true, postbreak=\mbox{\textcolor{red}{$\hookrightarrow$}\space}]
        
    \end{lstlisting}
\end{minipage}%