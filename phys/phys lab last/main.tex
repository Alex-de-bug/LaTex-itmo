\documentclass[landscape]{article}
\usepackage[utf8]{inputenc} %кодировка
\usepackage[T2A]{fontenc}
\usepackage[english,russian]{babel} %русификатор 
\usepackage{mathtools} %библиотека матеши
\usepackage[left=0cm,right=0.5cm,top=2cm,bottom=2cm,bindingoffset=0cm]{geometry} %изменение отступов на листе
\usepackage{amsmath}
\usepackage{graphicx} %библиотека для графики и картинок
\graphicspath{}
\DeclareGraphicsExtensions{.pdf,.png,.jpg}
\usepackage{subcaption}
\usepackage{pgfplots}
\usepackage{graphicx} % Для изменения размера графика

\begin{document}
\begin{figure}[htbp]
    \centering
    \begin{tikzpicture}
    \begin{axis}[
        xlabel={$\varepsilon$ (рад/c$^2$)},
        ylabel={$M$ (Н$\cdot$м)},
        grid=both,
        width=\textwidth, 
        height=0.7\textheight,
        xmax=14,
        ymax=0.25,
        mark=*,
        only marks,
        xtick={0,0.4,...,14}, 
        ytick={0,0.01,...,0.25}, 
    ]


    \addplot+[
        error bars/.cd,
        x dir=both, x explicit,
        y dir=both, y explicit,
    ] 
        coordinates {
        (0.48363, 0.060113)   +- (0.09, 0.0007)
        (1.32820, 0.109427)   +- (0.09, 0.0007)
        (1.96525, 0.158622)   +- (0.09, 0.0007)
        (2.88792, 0.207529)   +- (0.09, 0.0007)
        (1.218103627, 0.0600097)   +- (0.09, 0.0007)
        (2.386177717, 0.1091550)   +- (0.09, 0.0007)
        (3.445184907, 0.1580692)   +- (0.09, 0.0007)
        (5.040696876, 0.2064739)   +- (0.09, 0.0007)
        (2.205338, 0.059870311)   +- (0.09, 0.0007)
        (4.224232, 0.108681541)   +- (0.09, 0.0007)
        (6.184133, 0.157044915)   +- (0.09, 0.0007)
        (9.290377, 0.204389957)   +- (0.09, 0.0007)
    };
    % Аппроксимирующая линия (линейная)
    \addplot[smooth] coordinates {
        (0.48363, 0.060113)
        (2.82792, 0.207529) 
    };
    % Аппроксимирующая линия (линейная)
    \addplot[smooth] coordinates {
        (1.218103627, 0.0600097)
        (4.840696876, 0.2064739)
    };
    % Аппроксимирующая линия (линейная)
    \addplot[smooth] coordinates {
        (2.205338, 0.059870311)
        (9.290377, 0.210389957)
    };

    \addplot+[
        error bars/.cd,
        x dir=both, x explicit,
        y dir=both, y explicit
    ] 
        coordinates {
        (0.964057, 0.060045)   +- (0.09, 0.0007)
        (1.564429, 0.109367)   +- (0.09, 0.0007)
        (2.724521, 0.158338)   +- (0.09, 0.0007)
        (3.472116, 0.207243)   +- (0.09, 0.0007)
        (1.565423755, 0.0599606)   +- (0.09, 0.0007)
        (2.824003255, 0.1090422)   +- (0.09, 0.0007)
        (4.850889122, 0.1575435)   +- (0.09, 0.0007)
        (5.748639868, 0.2061267)   +- (0.09, 0.0007)
        (2.8, 0.0597855)   +- (0.09, 0.0007)
        (5.4, 0.1083626)   +- (0.09, 0.0007)
        (9.0, 0.1559736)   +- (0.09, 0.0007)
        (13.3, 0.2024037)  +- (0.09, 0.0007)
    };
    % Аппроксимирующая линия (линейная)
    \addplot[smooth] coordinates {
        (0.884057, 0.060045)
        (3.492116, 0.207243)
    };
     % Аппроксимирующая линия (линейная)
     \addplot[smooth] coordinates {
        (1.565423755, 0.0599606)
        (5.988639868, 0.2061267)
    };
    % Аппроксимирующая линия (линейная)
    \addplot[smooth] coordinates {
        (2.8, 0.0619606)
        (13.3, 0.2074037)
    };

    \end{axis}
    \end{tikzpicture}
    \caption{График зависимости M от $\varepsilon$ для разных положений утяжелителей}
    \label{fig:my_graph}
\end{figure}

\begin{figure}[htbp]
    \centering
    \begin{tikzpicture}
    \begin{axis}[
        xlabel={$R^2$},
        ylabel={$I$},
        grid=both,
        width=0.8\textwidth,
        height=0.5\textwidth,
        xmax=0.02,
        ymax=0.07,
        legend pos=north west,
        mark=*,
        only marks,
        xtick={0,0.005,...,0.02},
        ytick={0,0.01,...,0.07}, 
    ]
    \addplot+[
        only marks,
        mark=*,
        mark options={fill=white},
    ] 
        coordinates {
        (0.002, 0.013)
        (0.005, 0.020)
        (0.007, 0.032)
        (0.010, 0.039)
        (0.012, 0.056)
        (0.015, 0.056)
    };
    % Аппроксимирующая линия (линейная)
    \addplot[smooth, blue] coordinates {
        (0.002, 0.011)
        (0.015, 0.060)
    };
    
    \legend{$I(R2)$}
    \end{axis}
    \end{tikzpicture}
    \caption{График зависимости $I(R^2)$}
    \label{fig:I_R2_graph}
\end{figure}
\end{document}

