\documentclass{article}
\usepackage[utf8]{inputenc} %кодировка
\usepackage[T2A]{fontenc}
\usepackage[english,russian]{babel} %русификатор 
\usepackage{mathtools} %библиотека матеши
\usepackage[left=1cm,right=1cm,top=2cm,bottom=2cm,bindingoffset=0cm]{geometry} %изменение отступов на листе
\usepackage{amsmath}
\usepackage{graphicx} %библиотека для графики и картинок
\graphicspath{}
\DeclareGraphicsExtensions{.pdf,.png,.jpg}
\usepackage{subcaption}
\usepackage{pgfplots}

\begin{document}
\begin{tikzpicture}
    \begin{axis}[
        xlabel={$x_i$},
        ylabel={$n_i$},
        xtick={56.5,61.5,66.5,71.5,76.5,81.5,86.5,91.5,96.5},
        ytick={8,9,10,11,12,14,15},
        xmin=55, xmax=100,
        ymin=7, ymax=16,
        grid=both,
        width=10cm,
        height=7cm,
        legend pos=north west,
        legend style={cells={anchor=west}},
        legend entries={Полигон частот},
    ]
    
    \addplot+[
        mark=*,
        sharp plot,
        ] coordinates {
        (56.5, 9)
        (61.5, 10)
        (66.5, 11)
        (71.5, 14)
        (76.5, 15)
        (81.5, 10)
        (86.5, 12)
        (91.5, 11)
        (96.5, 8)
    };
    
    \end{axis}
    \end{tikzpicture}



    \begin{tikzpicture}
        \begin{axis}[
            xlabel={$x_i$},
            ylabel={$\frac{W_i}{h}$},
            xtick=data,
            symbolic x coords={{[54-59)},{[59-64)},{[64-69)},{[69-74)},{[74-79)},{[79-84)},{[84-89)},{[89-94)},{[94-99)},{[99-104]}},
            ymin=0, ymax=0.04,
            ytick={0.02,0.022,0.028,0.03,0.02,0.024,0.022,0.018,0.016},
            grid=both,
            width=14cm,
            height=8cm,
            bar width=20pt,
            legend pos=north east,
            ybar interval,
            scaled y ticks=false,
            tickwidth=0,
            axis on top,
            yticklabel style={/pgf/number format/.cd,fixed,precision=3},
            legend pos=north west,
            legend style={cells={anchor=west}},
            legend entries={Гистограмма относительных частот},
        ]

        \addplot+[fill=blue] coordinates {
                ({[54-59)}, 0.018)
                ({[59-64)}, 0.02)
                ({[64-69)}, 0.022)
                ({[69-74)}, 0.028)
                ({[74-79)}, 0.03)
                ({[79-84)}, 0.02)
                ({[84-89)}, 0.024)
                ({[89-94)}, 0.022)
                ({[94-99)}, 0.016)
                ({[99-104]}, 0.01)
            };
        

        
        \end{axis}
        \end{tikzpicture}

        \begin{tikzpicture}
            \begin{axis}[
                legend pos=north west,
                legend style={cells={anchor=west}},
                legend entries={Эмпирическая функция распределения},
                xlabel={x},
                ylabel={$F^*(x)$},
                width=12cm,
                height=8cm,
                grid=both,
                ymin=0,
                ymax=1,
                xmin=54,
                xmax=99,
                % ytick={0,0.1,0.2,0.3,0.4,0.5,0.6,0.7,0.8,0.9,1},
                ytick={0.09,0.19,0.3,0.44,0.59,0.69,0.81,0.92,1},
                xtick={54,59,64,69,74,79,84,89,94,99},
                extra x ticks={54,59,64,69,74,79,84,89,94,99},
            ]
            
            \addplot+[smooth] coordinates {
                (54, 0)
                (59, 0.09)
                (64, 0.19)
                (69, 0.3)
                (74, 0.44)
                (79, 0.59)
                (84, 0.69)
                (89, 0.81)
                (94, 0.92)
                (99, 1)
            };
            
            \end{axis}
        \end{tikzpicture}

        \begin{tikzpicture}
            \begin{axis}[
                xlabel=$x$,
                ylabel=$y$,
                legend pos=outer north east,
                xmin=0, xmax=1290, % установите необходимые пределы для x
                ymin=0, ymax=1.7,   % установите необходимые пределы для y
                grid=major,
                width=10cm,
                height=8cm
            ]
            
            % Точки
            \addplot[only marks, mark=*] coordinates {
                (250, 0.2)
                (250, 0.4)
                (250, 0.6)

                (450, 0.4)
                (450, 0.6)
                (450, 0.8)

                (650, 0.8)
                (650, 1.0)
                (650, 1.2)

                (850, 0.8)
                (850, 1.0)
                (850, 1.2)

                (1050, 1)
                (1050, 1.2)
                (1050, 1.4)

                (1250, 1.4)
                (1250, 1.6)
            };
            
            % Линия регрессии
            \addplot[blue, domain=0:1260, samples=100] {0.00094*x + 0.270218};
            \addlegendentry{Очень случайные точки}
            \addlegendentry{Линия регрессии}
            
            \end{axis}
            \end{tikzpicture}

\end{document}
