\documentclass{article}
\usepackage[utf8]{inputenc} %кодировка
\usepackage[T2A]{fontenc}
\usepackage[english,russian]{babel} %русификатор 
\usepackage{mathtools} %библиотека матеши
\usepackage[left=1cm,right=1cm,top=1cm,bottom=0cm,bindingoffset=0cm]{geometry} %изменение отступов на листе
\usepackage{amsmath}
\usepackage{graphicx} %библиотека для графики и картинок
\graphicspath{}
\DeclareGraphicsExtensions{.pdf,.png,.jpg}
\usepackage{subcaption}
\usepackage{pgfplots}
\usepackage{xcolor}
\usepackage{amsmath}
\usepackage{graphicx} %библиотека для графики и картинок
\graphicspath{}
\DeclareGraphicsExtensions{.pdf,.png,.jpg}
\usepackage{subcaption}
\usepackage{pgfplots}
\usepackage{amssymb}
\usepackage{physics}
\usepackage{listings}
\usepackage{fancyvrb}
% \newcommand{\bil}[2]{%
% \begin{minipage}[t]{0.3\textwidth}
%     \textbf{#1} \\
%     \begin{lstlisting}[language=Java, frame=single, basicstyle=\tiny\ttfamily, breaklines=true, breakatwhitespace=true, postbreak=\mbox{\textcolor{red}{$\hookrightarrow$}\space}]
%         #2
%     \end{lstlisting}
% \end{minipage}%
% }
\lstset{basicstyle=\tiny\ttfamily}

% Define JavaScript language



\begin{document}
\begin{center}
    \LaTeX
\end{center}
\tiny


\begin{minipage}[t]{0.3\textwidth}
    \textbf{Написать исходный код CDI бина, реализующего паттерн «команда» (Command Pattern)}
    \begin{lstlisting}[frame=single, basicstyle=\tiny\ttfamily, breaklines=true, breakatwhitespace=true, postbreak=\mbox{\textcolor{red}{$\hookrightarrow$}\space}]
interface Command {void execute();}
@Named(value="cmd1") @ApplicationScoped
public class Cmd1 implements Command {
    void execute() { ... };}
@Named(value="cmd2") @ApplicationScoped
public class Cmd2 implements Command {
    void execute() { ... };}
@Named @ApplicationScoped
public class MyBean implements Command {
    private final Command cmd1, cmd2;
    @Inject
    public MyBean(@Named("cmd1") Command cmd1, @Named("cmd2") Command cmd2) { this.cmd1 = cmd1; this.cmd2 = cmd2;}
    public void cmd(int n) {if (n == 1) cmd1.execute(); if (n == 2) cmd2.execute();}}
    \end{lstlisting}
\end{minipage}%
\hfill
\begin{minipage}[t]{0.3\textwidth}
    \textbf{JSF страница, динамически подгружаемая и выводящая новостную ленту с новостями формата: автор, заголовок, дата, иллюстрация, аннотация и полный текст(показывается при нажатии на соответствующую строчку)}
    \begin{lstlisting}[frame=single, basicstyle=\tiny\ttfamily, breaklines=true, breakatwhitespace=true, postbreak=\mbox{\textcolor{red}{$\hookrightarrow$}\space}]
<h:body> <h:dataTable value="#{newsBean.newsList}" var="news" border="1">
<h:column> <f:facet name="header">Author</f:facet>
<h:outputText value="#{news.author}" /> </h:column>
<h:column> <f:facet name="header">Title</f:facet>
<h:outputText value="#{news.title}" /> </h:column>
<h:column> <f:facet name="header">Date</f:facet>
<h:outputText value="#{news.date}" /> </h:column>
<h:column> <f:facet name="header">Image</f:facet>
<h:graphicImage value="#{news.image}" width="100" height="100" alt="News Image" />
</h:column> <h:column> <f:facet name="header">Summary</f:facet>
<h:outputText value="#{news.summary}" /> </h:column>
<h:column> <f:facet name="header">Full Text</f:facet>
<h:commandLink value="Show Full Text" action="#{newsBean.showFullText(news)}">
<f:setPropertyActionListener target="#{newsBean.selectedNews}" value="#{news}" />
</h:commandLink> </h:column> </h:dataTable>
<h:panelGroup rendered="#{not empty newsBean.selectedNews}">
<h3>Full Text of News:</h3>
<h:outputText value="#{newsBean.selectedNews.fullText}" escape="false" />
</h:panelGroup> </h:body>
    \end{lstlisting}
\end{minipage}%
\hfill
\begin{minipage}[t]{0.3\textwidth}
    \textbf{Написать React компонент формирующий таблицу пользователей, данные приходят в props}
\begin{lstlisting}[frame=single, basicstyle=\tiny\ttfamily, breaklines=true, breakatwhitespace=true, postbreak=\mbox{\textcolor{red}{$\hookrightarrow$}\space}]
function UsersTable(props) { return (
<table className="table"> <thead>
<tr><td>Name</td><td>Surname</td></tr>
</thead> <tbody> { props.data.map((user, i) => (
<tr key={i}> <td>{user.name}</td>
<td>{user.surname}</td> </tr>))} </tbody> </table>)}
\end{lstlisting}
\end{minipage}%
\\

\begin{minipage}[t]{0.3\textwidth}
    \textbf{Написать веб-приложение на JSF (xhtml + CDI-бин) со списком студентов и бин, который будет реализовывать логику отчисления студентов. Напротив каждого имени студента должна быть кнопка "отчислить". Обновление должно производиться при помощи AJAX}
    \begin{lstlisting}[frame=single, basicstyle=\tiny\ttfamily, breaklines=true, breakatwhitespace=true, postbreak=\mbox{\textcolor{red}{$\hookrightarrow$}\space}]
@Named
@ApplicationScoped
public class StudentBean implements Serializable {
private List<String> students;
public List<String> getStudents(){ return students;}
public void expelStudent(String studentName) {
    students.remove(studentName);}}
    
<h:body><h:form>
<ui:repeat value="#{studentBean.students}" var="student"> #{student}
<h:commandButton value="Expel" action="#{stude ntBean.expelStudent(student)}">
<f:ajax execute="@this" render="@form" />
</h:commandButton>
<br/></ui:repeat></h:form></h:body>
    \end{lstlisting}
\end{minipage}%
\hfill
\begin{minipage}[t]{0.3\textwidth}
    \textbf{Интерфейс на React, формирующий две страницы с разными URL: Главную (/home) и Новости (/news). Переход между страницами должен осуществляться посредством гиперссылок.}
    \begin{lstlisting}[ frame=single, basicstyle=\tiny\ttfamily, breaklines=true, breakatwhitespace=true, postbreak=\mbox{\textcolor{red}{$\hookrightarrow$}\space}]
export function App(props) {
return (<BrowserRouter>
<Routes><Route path="/home" element={<div><h1>Home</h1> <a href='/news' > news</a></div>} />
<Route path="/news" element={<div><h1>news</h1> <a href='/home' > home</a></div>} />
</Routes></BrowserRouter>);}
    \end{lstlisting}
\end{minipage}%
\hfill
\begin{minipage}[t]{0.3\textwidth}
    \textbf{Написать CDI Bean калькулятор, поддерживающий 4 базовые операции для целых чисел}
    \begin{lstlisting}[frame=single, basicstyle=\tiny\ttfamily, breaklines=true, breakatwhitespace=true, postbreak=\mbox{\textcolor{red}{$\hookrightarrow$}\space}]
@Named(value="calc")
@ApplicationScoped
public class Calc implements Serializeable{
public int add(int a, int b) { return a + b; }
public int sub(int a, int b) { return a - b; }
public int mul(int a, int b) { return a * b; }
public int div(int a, int b) { return a / b; }}
    \end{lstlisting}
\end{minipage}%
\\

\begin{minipage}[t]{0.3\textwidth}
    \textbf{Интерфейс JSF (xhtml страница + CDI), реализующий ввод паспортных данных (серия, номер, дата, место выдачи)}
    \begin{lstlisting}[frame=single, basicstyle=\tiny\ttfamily, breaklines=true, breakatwhitespace=true, postbreak=\mbox{\textcolor{red}{$\hookrightarrow$}\space}]
@ManagedBean @SessionScoped @Setter @Getter 
public class PassportInputBean { 	
private Long series; private Long number; 	
private Date date; private String place; 	
@ManagedProperty(value = "#{repoBean}") 
private PassportRepo repo; 	
public void storePassportData() { 		
repo.store(series, number, date, place);}}
<h:form>Enter series	
<h:inputText type="number" value="#{passportInputBean.series}"
required="true"><f:validateLongRange minimum="0" maximum="9999"/>	
</h:inputText>
<h:inputText type="number" value="#{passportInputBean.number}"> 
<f:validateLongRange minimum="0" maximum="999999" />
</h:inputText>
<h:inputText value="#{passportInputBean.date}">
<f:convertDateTime pattern="yyyy-MM-dd"/> 	
</h:inputText> Enter born place 	
<h:inputText type="text" value="#{passportInputBean.place}">
<h:commandButton value="Send" action="#{passportInputBean.storePas
sportData()}"> </h:form>
    \end{lstlisting}
\end{minipage}%
\hfill
\begin{minipage}[t]{0.3\textwidth}
    \textbf{Привести фрагмент кода управляемого бина, увеличивающего на 1 значение, отображаемое на кнопке при каждом клике по ней}
    \begin{lstlisting}[frame=single, basicstyle=\tiny\ttfamily, breaklines=true, breakatwhitespace=true, postbreak=\mbox{\textcolor{red}{$\hookrightarrow$}\space}]
@ManagedBean @ApplicationScoped
public class MyBean implements Serializable {
private int value = 0; public void increment() {
value++;} public int getValue() {return value;}
public void setValue(int value) {this.value = value;}}
<h:commandButton action= "#{myBean.increment}" value = "#{bean.value}"/>
    \end{lstlisting}
\end{minipage}%
\hfill
\begin{minipage}[t]{0.3\textwidth}
    \textbf{Сделать бин который показывает время в минутах со старта сервера}
    \begin{lstlisting}[frame=single, basicstyle=\tiny\ttfamily, breaklines=true, breakatwhitespace=true, postbreak=\mbox{\textcolor{red}{$\hookrightarrow$}\space}]
@Named("serverTimer") @ApplicaitonScoped
public class ServerTimer { private long start;
public ServerTimer() {start = System.currentTimeMillis();}
public double getTime() {long currentTime = System.currentTimeMillis();
return Math.floor((currentTime - start) / 60000.0);}}
    \end{lstlisting}
\end{minipage}%
\\ 

\begin{minipage}[t]{0.3\textwidth}
    \textbf{Написать страницу JSF, которая бы выводила сначала 10 простых чисел, а затем с помощью Ajax запроса ещё 10.}
    \begin{lstlisting}[frame=single, basicstyle=\tiny\ttfamily, breaklines=true, breakatwhitespace=true, postbreak=\mbox{\textcolor{red}{$\hookrightarrow$}\space}]
@ApplicationScoped @Named public class PrimeNumber{@Getter
private final List<Long> primes = new ArrayList<>();
public PrimeNumber(){nextPrimes();}
public nextPrimes(){//add in primes next 10 prime numbers}}

<h:dataTable id="table" value"#{primeNumber.primers}" var="prime">
<h:column> <h:outputText value="#{prime}"/>
</h:column> </h:dataTable> <h:form>
<h:commandButton value="Next 10" action="#{primeNumber.nextPrimes}">
<f:ajax execute="@form" render="table"/>
</h:commandButton> </h:form>
    \end{lstlisting}
\end{minipage}%
\hfill
\begin{minipage}[t]{0.3\textwidth}
    \textbf{JSF Manager Bean, после инициализации HTTP-сессии формирующий коллекцию с содержимым таблицы Н\_УЧЕБНЫЕ\_ПЛАНЫ. Для доступа к БД необходимо использовать JDBC-ресурс jdbc/OrbisPool.}
    \begin{lstlisting}[frame=single, basicstyle=\tiny\ttfamily, breaklines=true, breakatwhitespace=true, postbreak=\mbox{\textcolor{red}{$\hookrightarrow$}\space}]
@Named("myBean") @SessionScoped class MyBean {
private List<String> plans;
private List<String> getPlans() {return plans;}
@Resource(name="jdbc/OrbisPool",type=DataSource.class)
private DataSource dataSource; @PostConstruct
private void loadPlans() {
try (var conn = dataSource.getConnection()) {
var stmt = conn.createStatement();
var rs = stmt.executeQuery("SELECT name FROM plans");
plans = new ArrayList<>(); while (rs.next()) {
plans.add(rs.getString("name"));}}}}
    \end{lstlisting}
\end{minipage}%
\hfill
\begin{minipage}[t]{0.3\textwidth}
    \textbf{Написать managed bean и задать ему scope такой же как у бина otherBean}
    \begin{lstlisting}[frame=single, basicstyle=\tiny\ttfamily, breaklines=true, breakatwhitespace=true, postbreak=\mbox{\textcolor{red}{$\hookrightarrow$}\space}]
@ManagedBean @ApplicationScoped
public class OtherBean {
    @ManagedProperty(value="#{myBean}")
@Getter private MyBean myBean;}
@ManagedBean @NoneScoped @Named
public class MyBean{}
    \end{lstlisting}
\end{minipage}%
\\

\begin{minipage}[t]{0.3\textwidth}
    \textbf{Написать на JSF правило навигации в faces-config.xml со страницы page1.jsp на page2.jsp с использованием этой кнопки: <h:commandButton action="goToPage" value="Go to Page 2"/>}
    \begin{lstlisting}[frame=single, basicstyle=\tiny\ttfamily, breaklines=true, breakatwhitespace=true, postbreak=\mbox{\textcolor{red}{$\hookrightarrow$}\space}]
<navigation-rule>
<from-view-id>/page1.jsp</from-view-id>
<navigation-case>
<from-outcome>goToPage</from-outcome>
<to-view-id>/page2.jspl</to-view-id>
<redirect/></navigation-case></navigation-rule>
    \end{lstlisting}
\end{minipage}%
\hfill
\begin{minipage}[t]{0.3\textwidth}
    \textbf{RestController, который реализует перевод градусов Цельсия в Фаренгейты и обратно}
    \begin{lstlisting}[frame=single, basicstyle=\tiny\ttfamily, breaklines=true, breakatwhitespace=true, postbreak=\mbox{\textcolor{red}{$\hookrightarrow$}\space}]
@RestController class TempController {
@GetMapping("/convert/c/f")
double cToF(@RequestParam double c) {return c * 1.8 + 32.0;}
@GetMapping("/convert/f/c")
double fToC(@RequestParam double f) {return (f - 32.0) / 1.8;}}
    \end{lstlisting}
\end{minipage}%
\hfill
\begin{minipage}[t]{0.3\textwidth}
    \textbf{Написать JSF страницу с многострочным полем, в которое можно вводить только строчные символы латиницы.}
    \begin{lstlisting}[frame=single, basicstyle=\tiny\ttfamily, breaklines=true, breakatwhitespace=true, postbreak=\mbox{\textcolor{red}{$\hookrightarrow$}\space}]
<h:head><script>function validateInput(event) {
var textArea = document.getElementById('inputTextArea');
var regex = /^[a-z\s]*$/;
if (!regex.test(textArea.value)) {event.preventDefault();}}
</script></h:head><h:body><h:form>
<h:inputTextarea id="inputTextArea" rows="5" cols="30" oninput="validateInput(event)"/>
</h:form></h:body>
    \end{lstlisting}
\end{minipage}%
\\

\begin{minipage}[t]{0.3\textwidth}
    \textbf{Конфигурация faces-config, задающая managed bean с именем myBean, которым будет управлять сам программист}
    \begin{lstlisting}[frame=single, basicstyle=\tiny\ttfamily, breaklines=true, breakatwhitespace=true, postbreak=\mbox{\textcolor{red}{$\hookrightarrow$}\space}]
<faces-config version="2.2" ...> <managed-bean>
<managed-bean-name>myBean</managed-bean-name>
<managed-bean-class>MyBean</managed-bean-class>
<managed-bean-scope>custom</managed-bean-scope>
</managed-bean> </faces-config>
    \end{lstlisting}
\end{minipage}%
\hfill
\begin{minipage}[t]{0.3\textwidth}
    \textbf{Angular компонент, который позволяет поделиться чем-то в VK, Twitter, Facebook (API для соцсетей можно описать текстом)
    API принимает логин и пароль, а также сообщение, которое будет опубликовано на личной странице после прохождения авторизации.}
    \begin{lstlisting}[frame=single, basicstyle=\tiny\ttfamily, breaklines=true, breakatwhitespace=true, postbreak=\mbox{\textcolor{red}{$\hookrightarrow$}\space}]
import { Component} from '@angular/core';
import {HttpClient} from '@angular/common/http';
@Component({ selector: 'my-app',
template: <div class="form-group">
<label>Login</label>
<input class="form-control" name="username" [(ngModel)]="username" /> </div>
<div class="form-group"><label>Password</label>
<input class="form-control" name="password" [(ngModel)]="password" /></div>
<div class="form-group"><label>Message</label>
<input class="form-control" name="message" [(ngModel)]="message" /></div>
<div class="form-group">
<button class="btn btn-default" (click)="submit()">Send</button></div>})
export class AppComponent {username: string="";
password: string="";message: string="";
http: HttpClient;
submit(){const body = {login: username, password: password, message: message};
this.http.post('link to API', body);}}
    \end{lstlisting}
\end{minipage}%
\hfill
\begin{minipage}[t]{0.3\textwidth}
    \textbf{Приложение на Angular, реализующее форму для заполнения бланка на отчисление по собственному желанию. Форма должна принимать на вход имя пользователя и дату, и формировать заполненный бланк заявления (на клиентской стороне)}
    \begin{lstlisting}[frame=single, basicstyle=\tiny\ttfamily, breaklines=true, breakatwhitespace=true, postbreak=\mbox{\textcolor{red}{$\hookrightarrow$}\space}]
@Component({selector: 'app',template: `<form>
<input type="text" name="firstName" placeholder="Name" [(ngModel)]="firstName">
<input type="text" name="lastName" placeholder="lastName" [(ngModel)]="lastName">
<input type="date" name="lastName" [(ngModel)]="date"></form>
<main><h1>PSJ</h1>
<h2>{{ firstName }} {{ lastName }}, {{ date }}, PSJ</h2>
</main>`})
class AppComponent {firstName = "" lastName = ""
date = "01-09-2021"}
    \end{lstlisting}
\end{minipage}%
\\

\begin{minipage}[t]{0.3\textwidth}
    \textbf{Написать на Angular интерфейс, который проверяет если ли в куки sessionid и если нет, отправляет пользователя на аутентификацию по логину и паролю}
    \begin{lstlisting}[frame=single, basicstyle=\tiny\ttfamily, breaklines=true, breakatwhitespace=true, postbreak=\mbox{\textcolor{red}{$\hookrightarrow$}\space}]
@Component({selector: 'app',
template: `<form *ngIf="!hasCookie">
<h1>auth</h1><input type="text" name="username" placeholder="name">
<input type="password" name="password" placeholder="password">
<input type="submit" value="sing in"></form>`})
class AppComponent {get hasCookie(): boolean {
return document.cookie.indexOf("jSessionid=") != -1;}}
    \end{lstlisting}
\end{minipage}%
\hfill
\begin{minipage}[t]{0.3\textwidth}
    \textbf{Компонент для React, формирующий строку с автодополнением. Массив значений для автодополнения должен получаться с сервера посредством запроса к REST API}
    \begin{lstlisting}[frame=single, basicstyle=\tiny\ttfamily, breaklines=true, breakatwhitespace=true, postbreak=\mbox{\textcolor{red}{$\hookrightarrow$}\space}]
function Autocomplete() {
const [candidates, setCandidates] = useState([])
async function updateList(e) {
let res = await fetch("http://api.itmo.ru/stu
dents/search?q=" + e.target.value);
setCandidates(await res.json())} return <div>
<input type="text" onChange={updateList}>
<ul>{candidates.map((text, i) => <li key={i}>{text}</li>)}</ul></div>;}
    \end{lstlisting}
\end{minipage}%
\hfill
\begin{minipage}[t]{0.3\textwidth}
    \textbf{Реализовать бронь авиабилетов на jsf}
    \begin{lstlisting}[frame=single, basicstyle=\tiny\ttfamily, breaklines=true, breakatwhitespace=true, postbreak=\mbox{\textcolor{red}{$\hookrightarrow$}\space}]
@ManagedBean @SessionScoped @Getter @Setter
public class TicketBean implements Serializable {
@ManagedProperty(value = "#{repoBean}")
private TicketRepo repo;
private List<Ticket> available;
@PostConstruct public void init() {
available = repo.getAvailable();}
public void book(int id) { repo.book(id);
available = repo.getAvailable();}}
@Getter @Setter public class Ticket {
private int id; private int price;
private String source; private String dest;}
<h1>Book avia ticket></h1>
<ui:repeat value="#{tickets.available}" var="ticket">
<div class="ticket">
#{ticket.source} - #{ticket.destination}
Number: #{ticket.id} Cost: #{ticket.price} e.g.
<h:commandButton action="#{tickets.book(ticket.id)}"
 value="book"/></div></ui:repeat>
    \end{lstlisting}
\end{minipage}%
\\

\begin{minipage}[t]{0.3\textwidth}
    \textbf{REST - контроллер на Spring Web MVC, предоставляющий CRUD-интерфейс к таблице со списком покемонов}
    \begin{lstlisting}[frame=single, basicstyle=\tiny\ttfamily, breaklines=true, breakatwhitespace=true, postbreak=\mbox{\textcolor{red}{$\hookrightarrow$}\space}]
@RestController class PokemonResource {
@Autowired private PokemonRepository repository;
@GetMapping("/pokemons") List<Pokemon> all() {
return repository.findAll()}
@GetMapping("/pokemons/{id}")
Pokemon one(@PathVariable Long id) {
return repository.findOne(id)}
@PostMapping("/pokemons")
Long createNew(@RequestBody Pokemon pokemon) {
return repository.save(pokemon)}
@DeleteMapping("/pokemons/{id}")
void delete(@PathVariable Long id) {
repository.delete(id)}}
    \end{lstlisting}
\end{minipage}%
\hfill
\begin{minipage}[t]{0.3\textwidth}
    \textbf{Написать интерфейс для ввода данных банковской карты на React}
    \begin{lstlisting}[frame=single, basicstyle=\tiny\ttfamily, breaklines=true, breakatwhitespace=true, postbreak=\mbox{\textcolor{red}{$\hookrightarrow$}\space}]
export function App() {
const [name, setName] = React.useState('');
const [number, setNumber] = React.useState('');
function handleSubmit(event) {
event.preventDefault(); console.log('name:', name);
console.log('number:', number);}
return (<form onSubmit={handleSubmit}><div>
<label>Name</label><input type="text" maxLength={5}
value={name} onChange={(e)=>setName(e.target.value)}
/></div><div><label>CVS Number</label>
<input type="number" maxLength={8} value={number}
onChange={(e) => setNumber(e.target.value)}/></div>
<button type="submit">Submit</button></form>);}
    \end{lstlisting}
\end{minipage}%
\hfill
\begin{minipage}[t]{0.3\textwidth}
    \textbf{JSF страничка с данными из бина}
    \begin{lstlisting}[frame=single, basicstyle=\tiny\ttfamily, breaklines=true, breakatwhitespace=true, postbreak=\mbox{\textcolor{red}{$\hookrightarrow$}\space}]
@Named("myBean") @ManagedBean class MyBean {
int value; public int getValue(){return value;}}
<h:outputText value="#{myBean.value}"/>        
    \end{lstlisting}
\end{minipage}%
\\

\begin{minipage}[t]{0.3\textwidth}
    \textbf{Конфигурация, чтобы JSF обрабатывал все запросы приходящие с .xhtml и со всех URL, начинающихся с /faces/}
    \begin{lstlisting}[frame=single, basicstyle=\tiny\ttfamily, breaklines=true, breakatwhitespace=true, postbreak=\mbox{\textcolor{red}{$\hookrightarrow$}\space}]
<web-app><servlet>
<servlet-name>Faces Servlet</servlet-name>
<servlet-class>javax.faces.webapp.FacesServlet
</servlet-class></servlet><servlet-mapping>
<servlet-name>Faces Servlet</servlet-name>
<url-pattern>*.xhtml</url-pattern></servlet-mapping>
<servlet-mapping>
<servlet-name>Faces Servlet</servlet-name> 
<url-pattern>/faces/*</url-pattern>
</servlet-mapping></web-app>
    \end{lstlisting}
\end{minipage}%
\hfill
\begin{minipage}[t]{0.3\textwidth}
    \textbf{Интерфейс на Angular, который выводит интерактивные часы с обновление каждую секунду}
    \begin{lstlisting}[frame=single, basicstyle=\tiny\ttfamily, breaklines=true, breakatwhitespace=true, postbreak=\mbox{\textcolor{red}{$\hookrightarrow$}\space}]
@Component({ selector:'my-app', 
template: <div>{{time}}</div>})
export class AppComponent{
time: ""
setInterval(()=>this.time = new Date(),1000)}
    \end{lstlisting}
\end{minipage}%
\hfill
\begin{minipage}[t]{0.3\textwidth}
    \textbf{Интерфейс реализации логин+пароль на React. На стороне сервера- Rest API}
    \begin{lstlisting}[frame=single, basicstyle=\tiny\ttfamily, breaklines=true, breakatwhitespace=true, postbreak=\mbox{\textcolor{red}{$\hookrightarrow$}\space}]
function App(props) {
const [login, setLogin] = useState("")
const [password, setPassword] = useState("")
async function onSubmit() {
let res = await fetch("api/login", {
headers: {"Content-Type": "application/json"},
body: JSON.stringify({login: login, password: password})})
let token = await res.text();
localStorage.setItem("token", token)}
return (<form onSubmit={onSubmit}>
<input type="text" placeholder="login" required value={login}
onChange={e => setLogin(e.target.value)}/>
<input type="password" placeholder="Pass" required value={password}
onChange={e => setPassword(e.target.value)}/>
<input type="submit" value="Sign in"/>
</form>);}
    \end{lstlisting}
\end{minipage}%



\end{document}





\begin{minipage}[t]{0.3\textwidth}
    \textbf{}
    \begin{lstlisting}[frame=single, basicstyle=\tiny\ttfamily, breaklines=true, breakatwhitespace=true, postbreak=\mbox{\textcolor{red}{$\hookrightarrow$}\space}]
        
    \end{lstlisting}
\end{minipage}%